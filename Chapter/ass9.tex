\chapter

\problem

We start with metric
%
\begin{equation}
    \dd s^2 = -\dd t ^2 + a^2(t)\dd \bs{x}^2
\end{equation}
%
and we want to change coordinates $\bs{x} \to \bs{x}_p = a(t) \bs{x}$. For that
we first need to calculate
%
\begin{equation}
    \dd \bs{x} = \dd \left(\frac{\bs{x}_p}{a}\right) =
    \frac{\dd \bs{x}_p a - \frac{\dd a}{\dd t} \dd t ~\bs{x}_p}{a^2}=
    \frac{a \dd \bs{x}_p - \dot{a}\bs{x}_p \dd t}{a^2} =
    \frac{\dd \bs{x}_p}{a} - \frac{\dot{a}}{a^2}\bs{x}_p \dd t.
\end{equation}
%
Using that result I can now calculate term which is in the equation for line
segment
%
\begin{equation}
    a^2\dd x^2 = (a \dd x)^2 =
    \left(\dd \bs{x}_p - \frac{\dot{a}}{a}\bs{x}_p \dd t\right)^2=
    \dd \bs{x}_p^2 -
    2\frac{\dot{a}}{a}\bs{x}_p \dd t \dd \bs{x}_p +
    \left(\frac{\dot{a}}{a}\right)^2\bs{x}_p^2 \dd t^2.
\end{equation}
%
Substituting this into metric gives
%
\begin{multline}
    \dd s^2 = -\dd t ^2 + a^2(t)\dd \bs{x}^2 =
    -\dd t ^2 +
    \dd \bs{x}_p^2 -
    2\frac{\dot{a}}{a}\bs{x}_p \dd t \dd \bs{x}_p +
    \left(\frac{\dot{a}}{a}\right)^2\bs{x}_p^2 \dd t^2 = \\
    \boxed{-\left(1 - \left(\frac{\dot{a}}{a}\right)^2\bs{x}_p^2\right) \dd t^2 +
        \dd \bs{x}_p^2 -
        2\frac{\dot{a}}{a}\bs{x}_p~\dd t \dd \bs{x}_p}
\end{multline}

\problem

Line segment in 4D Euclidian space is given by
%
\begin{equation}
    \dd s^2 = \dd w^2 + \dd x^2 + \dd y^2 + \dd z^2
\end{equation}
%
and 4D sphere of radius $a$ can be parametrize by
%
\begin{subequations}
    \begin{align}
        w = & a \cos \chi                       \\
        x = & a \sin \chi \sin \theta \cos \phi \\
        y = & a \sin \chi \sin \theta \sin \phi \\
        z = & a \sin \chi \cos \theta.
    \end{align}
\end{subequations}
%
We can calculate
%
\begin{subequations}
    \begin{align}
        \dd w = & - a \sin \chi \dd \chi                       \\
        \dd x = & a \cos \chi \sin \theta \cos \phi \dd \chi +
        a \sin \chi \cos \theta \cos \phi \dd \theta -
        a \sin \chi \sin \theta \sin \phi \dd \phi             \\
        \dd y = & a \cos \chi \sin \theta \sin \phi \dd \chi +
        a \sin \chi \cos \theta \sin \phi \dd \theta +
        a \sin \chi \sin \theta \cos \phi \dd \phi             \\
        \dd z = & a \cos \chi \cos \theta \dd \chi -
        a \sin \chi \sin \theta \dd \theta.
    \end{align}
\end{subequations}
%
and furthermore
%
\begin{align*}
    \dd x^2 + \dd y^2 = a^2( & \cos^2 \chi \sin^2 \theta \dd \chi^2 +                             \\
                             & \sin^2 \chi \cos^2 \theta \dd \theta^2 +                           \\
                             & \sin^2 \chi \sin^2 \theta \dd \phi^2 +                             \\
                             & 2\cos \chi \sin \chi \cos \theta \sin \theta \dd \chi \dd \theta )
\end{align*}
%
and
%
\begin{align*}
    \dd x^2 + \dd y^2 + \dd z^2 = a^2( & \cos^2 \chi \dd \chi^2 +              \\
                                       & \sin^2 \chi \dd \theta^2 +            \\
                                       & \sin^2 \chi \sin^2 \theta \dd \phi^2)
\end{align*}
%
and
%
\begin{align*}
    \dd w^2 + \dd x^2 + \dd y^2 + \dd z^2 = a^2( & \dd \chi^2 +                           \\
                                                 & \sin^2 \chi \dd \theta^2 +             \\
                                                 & \sin^2 \chi \sin^2 \theta \dd \phi^2).
\end{align*}
%
So eventually
%
\begin{equation}
    \boxed{\dd w^2 + \dd x^2 + \dd y^2 + \dd z^2 =
        a^2\left( \dd \chi^2 +
        \sin^2 \chi \left(\dd \theta^2 +
        \sin^2 \theta \dd \phi^2\right)\right)}
\end{equation}
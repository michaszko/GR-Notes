\chapter[9]

\problem

We start with metric
%
\begin{equation}
    \dd s^2 = -\dd t ^2 + a^2(t)\dd \bs{x}^2
\end{equation}
%
and we want to change coordinates $\bs{x} \to \bs{x}_p = a(t) \bs{x}$. For that
we first need to calculate
%
\begin{equation}
    \dd \bs{x} = \dd \left(\frac{\bs{x}_p}{a}\right) =
    \frac{\dd \bs{x}_p a - \frac{\dd a}{\dd t} \dd t ~\bs{x}_p}{a^2}=
    \frac{a \dd \bs{x}_p - \dot{a}\bs{x}_p \dd t}{a^2} =
    \frac{\dd \bs{x}_p}{a} - \frac{\dot{a}}{a^2}\bs{x}_p \dd t.
\end{equation}
%
Using that result I can now calculate term which is in the equation for line
segment
%
\begin{equation}
    a^2\dd x^2 = (a \dd x)^2 =
    \left(\dd \bs{x}_p - \frac{\dot{a}}{a}\bs{x}_p \dd t\right)^2=
    \dd \bs{x}_p^2 -
    2\frac{\dot{a}}{a}\bs{x}_p \dd t \dd \bs{x}_p +
    \left(\frac{\dot{a}}{a}\right)^2\bs{x}_p^2 \dd t^2.
\end{equation}
%
Substituting this into metric gives
%
\begin{multline}
    \dd s^2 = -\dd t ^2 + a^2(t)\dd \bs{x}^2 =
    -\dd t ^2 +
    \dd \bs{x}_p^2 -
    2\frac{\dot{a}}{a}\bs{x}_p \dd t \dd \bs{x}_p +
    \left(\frac{\dot{a}}{a}\right)^2\bs{x}_p^2 \dd t^2 = \\
    \boxed{-\left(1 - \left(\frac{\dot{a}}{a}\right)^2\bs{x}_p^2\right) \dd t^2 +
        \dd \bs{x}_p^2 -
        2\frac{\dot{a}}{a}\bs{x}_p~\dd t \dd \bs{x}_p}
\end{multline}

\problem

Line segment in 4D Euclidian space is given by
%
\begin{equation}
    \dd s^2 = \dd w^2 + \dd x^2 + \dd y^2 + \dd z^2
\end{equation}
%
and 4D sphere of radius $a$ can be parametrize by
%
\begin{subequations}
    \begin{align}
        w = & a \cos \chi                       \\
        x = & a \sin \chi \sin \theta \cos \phi \\
        y = & a \sin \chi \sin \theta \sin \phi \\
        z = & a \sin \chi \cos \theta.
    \end{align}
\end{subequations}
%
We can calculate
%
\begin{subequations}
    \begin{align}
        \dd w = & - a \sin \chi \dd \chi                       \\
        \dd x = & a \cos \chi \sin \theta \cos \phi \dd \chi +
        a \sin \chi \cos \theta \cos \phi \dd \theta -
        a \sin \chi \sin \theta \sin \phi \dd \phi             \\
        \dd y = & a \cos \chi \sin \theta \sin \phi \dd \chi +
        a \sin \chi \cos \theta \sin \phi \dd \theta +
        a \sin \chi \sin \theta \cos \phi \dd \phi             \\
        \dd z = & a \cos \chi \cos \theta \dd \chi -
        a \sin \chi \sin \theta \dd \theta.
    \end{align}
\end{subequations}
%
and furthermore
%
\begin{align*}
    \dd x^2 + \dd y^2 = a^2( & \cos^2 \chi \sin^2 \theta \dd \chi^2 +                             \\
                             & \sin^2 \chi \cos^2 \theta \dd \theta^2 +                           \\
                             & \sin^2 \chi \sin^2 \theta \dd \phi^2 +                             \\
                             & 2\cos \chi \sin \chi \cos \theta \sin \theta \dd \chi \dd \theta )
\end{align*}
%
and
%
\begin{align*}
    \dd x^2 + \dd y^2 + \dd z^2 = a^2( & \cos^2 \chi \dd \chi^2 +              \\
                                       & \sin^2 \chi \dd \theta^2 +            \\
                                       & \sin^2 \chi \sin^2 \theta \dd \phi^2)
\end{align*}
%
and
%
\begin{align*}
    \dd w^2 + \dd x^2 + \dd y^2 + \dd z^2 = a^2( & \dd \chi^2 +                           \\
                                                 & \sin^2 \chi \dd \theta^2 +             \\
                                                 & \sin^2 \chi \sin^2 \theta \dd \phi^2).
\end{align*}
%
So eventually
%
\begin{equation}
    \boxed{\dd w^2 + \dd x^2 + \dd y^2 + \dd z^2 =
        a^2\left( \dd \chi^2 +
        \sin^2 \chi \left(\dd \theta^2 +
        \sin^2 \theta \dd \phi^2\right)\right)}
\end{equation}

\problem

\subproblem

We have given two equations
%
\begin{subequations}
    \begin{align}
        \left(\frac{\dot{a}}{a}\right)^2 + \frac{k}{a^2}         & = \frac{8\pi G}{3} \rho \\
        \dot{\rho} + 3 \left(\frac{\dot{a}}{a}\right) (\rho + p) & = 0.
    \end{align}
\end{subequations}
%
First thing we do is to change derivative with respect to time to derivative
with respect to proper time given by $\dd \tau = \frac{\dd t}{a}$
%
\begin{equation}
    \dot{a} = \frac{\dd a}{\dd t} =
    \frac{\dd a}{\dd \tau} \frac{\dd \tau}{\dd t} =
    \frac{a'}{a}.
\end{equation}
%
Rewriting first equations
%
\begin{subequations}
    \begin{align}
        \frac{a'^{2}}{a^4} + \frac{k}{a^2}                         & = \frac{8\pi G}{3} \rho \\
        \frac{\rho'}{a} + 3 \left(\frac{a'}{a^2}\right) (\rho + p) & = 0.
        \label{eq:ass9_rho}
    \end{align}
\end{subequations}
%
Muliplying first equation by $a^4$ and taking derivative with respect to $\tau$
gives
%
\begin{equation}
    2 a' a'' + 2k a' a =
    4\frac{8\pi G}{3} a' a^3 \rho + \frac{8\pi G}{3} a^4 \rho'.
\end{equation}
%
Now I substitute second equation in place of $\rho'$
%
\begin{equation}
    2 a' a'' + 2k a' a =
    4\frac{8\pi G}{3} a' a^3 \rho - 3\frac{8\pi G}{3} a^4 \frac{a'}{a} (\rho + p).
\end{equation}
%
Now divide by $2a'$ and rearrange RHS
%
\begin{equation}
    a'' + k a =
    \frac{4\pi G}{3} a^3 \left(4 \rho - 3 \rho - 3p \right).
\end{equation}
%
And finally
%
\begin{equation}
    \boxed{a'' + k a =
        \frac{4\pi G}{3} \left(\rho - 3p \right) a^3.}
    \label{eq:ass9_sol}
\end{equation}

\subproblem

First I solve \cref{eq:ass9_rho} for $p=0$
%
\begin{equation}
    \rho' + 3 \frac{a'}{a}\rho  = 0
    \qquad\implies\qquad
    \frac{\rho'}{\rho} = -3 \frac{a'}{a}
    \qquad\implies\qquad
    \ln(\rho)' = -3 \ln(a)'.
\end{equation}
%
Integrating it
%
\begin{equation}
    \ln(\rho) = -3\ln(a)
    \qquad\implies\qquad
    \rho = a^{-3}.
\end{equation}
%
Now I solve \cref{eq:ass9_sol} in matter dominated universe $p=0$ and
$\rho=a^{-3}$
%
\begin{equation}
    a'' + ka =
    \frac{4\pi G}{3} \rho a^3.
    \qquad\implies\qquad
    a'' + ka - \frac{4\pi G}{3} = 0.
\end{equation}
%
I change variables $b = a - \frac{4\pi G}{3k}$ and rewrite above equation
%
\begin{equation}
    b'' + kb = 0.
\end{equation}
%
Solutions
%
\begin{equation}
    a = \begin{cases}
        A \sin \tau + B\cos\tau + \frac{4\pi G}{3} & \quad k=1  \\
        \frac{2\pi G}{3}\tau^2 + A\tau + B         & \quad k=0  \\
        A e^{\tau} + Be^{-\tau} - \frac{4\pi G}{3} & \quad k=-1
    \end{cases}
\end{equation}
%
Applying initial conditions $a(0)=0, a'(0)=0$
%
\begin{equation}
    a = \begin{cases}
        \frac{4\pi G}{3} \left(1 -  \cos\tau\right) & \quad k=1  \\
        \frac{2\pi G}{3}\tau^2                      & \quad k=0  \\
        \frac{4\pi G}{3} \left(\frac{e^{\tau} + e^{-\tau}}{2} - 1\right)=
        \frac{4\pi G}{3} \left(\cosh\tau - 1\right) & \quad k=-1
    \end{cases}
\end{equation}
%
Simplifying it
%
\begin{equation}
    \boxed{a(\tau) = \frac{8\pi G}{3}
        \begin{cases}
            \sin^2\left(\frac{\tau}{2}\right)  & \quad k=1  \\
            \left(\frac{\tau}{2}\right)^2      & \quad k=0  \\
            \sinh^2\left(\frac{\tau}{2}\right) & \quad k=-1
        \end{cases}}
\end{equation}
%
\begin{figure}[H]
    \centering
    \begin{tikzpicture}[domain=0:20]
        \pgfmathsetmacro{\g}{1.5}
        \pgfmathsetmacro{\gg}{1/\g}
        \begin{axis}
            [
                axis lines  = center,
                xlabel={$\tau$},
                ylabel={$a(\tau)$},
                xmin=0,
                xmax=6.5,
                ymin=0,
                ymax=6.5,
                ticks=none
            ]

            \addplot[domain = 0:6.28, samples = 1000,color=blue]
            gnuplot {sin(x/2)**2};
            \addplot[domain = 0:6.28, samples = 1000,color=red]
            gnuplot {(x/2)**2};
            \addplot[domain = 0:6.28, samples = 1000,color=green]
            gnuplot {sinh(x/2)**2};

            \node[inner sep=2pt,text=red] at (3.5,2) {$k=0$};
            \node[inner sep=2pt,text=green] at (1.2,1.5) {$k=-1$};
            \node[inner sep=2pt,text=blue] at (3.14,.5) {$k=1$};
        \end{axis}
    \end{tikzpicture}
    \caption{Matter dominated universe}
\end{figure}
%
Next I solve \cref{eq:ass9_sol} for radiation dominated universe
$p=\frac{1}{3}\rho$
%
\begin{equation}
    a'' + ka = 0
\end{equation}
%
Solutions
%
\begin{equation}
    a = \begin{cases}
        A \sin \tau + B\cos\tau & \quad k=1  \\
        A\tau + B               & \quad k=0  \\
        A e^{\tau} + Be^{-\tau} & \quad k=-1
    \end{cases}
\end{equation}
%
Applying initial conditions $a(0)=0, a'(0)=v$
%
\begin{equation}
    a = \begin{cases}
        v \sin\tau                       & \quad k=1  \\
        v \tau                           & \quad k=0  \\
        v~\frac{e^{\tau} - e^{-\tau}}{2} & \quad k=-1
    \end{cases}
\end{equation}
%
Simplifying it
%
\begin{equation}
    \boxed{a(\tau) = v\begin{cases}
            \sin\tau   & \quad k=1  \\
            \tau       & \quad k=0  \\
            \sinh \tau & \quad k=-1
        \end{cases}}
\end{equation}

\begin{figure}[H]
    \centering
    \begin{tikzpicture}[domain=0:20]
        \pgfmathsetmacro{\g}{1.5}
        \pgfmathsetmacro{\gg}{1/\g}
        \begin{axis}
            [
                axis lines  = center,
                xlabel={$\tau$},
                ylabel={$a(\tau)$},
                xmin=0,
                xmax=3.5,
                ymin=0,
                ymax=3.5,
                ticks=none
            ]

            \addplot[domain = 0:3.14, samples = 1000,color=blue]
            gnuplot {sin(x)};
            \addplot[domain = 0:3.14, samples = 1000,color=red]
            gnuplot {x};
            \addplot[domain = 0:3.14, samples = 1000,color=green]
            gnuplot {sinh(x)};

            \node[inner sep=2pt,text=red] at (2.5,2) {$k=0$};
            \node[inner sep=2pt,text=green] at (.5,1.5) {$k=-1$};
            \node[inner sep=2pt,text=blue] at (1.5,.5) {$k=1$};
        \end{axis}
    \end{tikzpicture}
    \caption{Radiation dominated universe}
\end{figure}
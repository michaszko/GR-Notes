\chapter[7]

\problem

Energy -- momentum tensor of perfect fluid
%
\begin{equation}
    T^{\mu\nu} = (\rho + p)\frac{u^\mu u^\nu}{c^2} + p g^{\mu\nu}.
\end{equation}
% 
First thing I do is base change
%
\begin{equation}
    \bs{T} = T^{\mu\nu}\partial_\mu\partial_\nu = T^{ab} \bs{e}_a\bs{e}_b =
    T^{ab} \bs{e}_a\bs{e}_b = T^{ab} e_a^\mu e_b^\nu \partial_\mu\partial_\nu.
\end{equation}
%
From this it can be seen that
%
\begin{equation}
    T^{\mu\nu} = T^{ab} e_a^\mu e_b^\nu.
\end{equation}
%
I multiply both sides by corresponding one-froms $\bar{e}_\mu^c \bar{e}_\nu^d$
which satisfy $\bar{e}_\mu^c e_d^\mu = \delta^c_d$
%
\begin{equation}
    T^{\mu\nu} \bar{e}^a_\mu \bar{e}^b_\nu = T^{ab}.
\end{equation}
%
From it I obtain
%
\begin{equation}
    T^{ab} = (\rho + p)\frac{u^\mu \bar{e}^a_\mu ~ u^\nu \bar{e}^b_\nu}{c^2} +
    p g^{\mu\nu}\bar{e}^a_\mu \bar{e}^b_\nu.
\end{equation}
%
Now I choose base $\{\bs{e}\}$ to be orthogonal (so $g_{\mu\nu}e_a^\mu e_b^\nu =
    \eta_{ab}$ and $g^{\mu\nu}\bar{e}^a_\mu \bar{e}^b_\nu = \eta^{ab}$) and
$u^\mu = c e^\mu_0$. Substituting those things yields
%
\begin{equation}
    T^{ab} = (\rho + p)\frac{\cancel{c} e^\mu_0 \bar{e}^a_\mu ~
        \cancel{c} e^\nu_0 \bar{e}^b_\nu}{\cancel{c^2}} +
    p \eta^{ab} =
    (\rho + p)\delta^a_0 ~ \delta^b_0 +
    p \eta^{ab}.
\end{equation}
%
So final result is
%
\begin{equation}
    T^{ab} \overset{*}{=}
    \begin{pmatrix}
        \rho & 0 & 0 & 0 \\
        0    & p & 0 & 0 \\
        0    & 0 & p & 0 \\
        0    & 0 & 0 & p
    \end{pmatrix}
\end{equation}

\problem

Calculate
%
\begin{equation}
    \partial_\mu T^{\mu\nu} =
    \partial_\mu \left\{(\rho + p)\frac{u^\mu u^\nu}{c^2} + p g^{\mu\nu}\right\}
\end{equation}
%
\begin{multline}
    \partial_\mu \left\{(\rho + p)\frac{u^\mu u^\nu}{c^2} + p g^{\mu\nu}\right\} =
    \partial_\mu (\rho + p)\frac{u^\mu u^\nu}{c^2} +
    (\rho + p)\partial_\mu\frac{u^\mu u^\nu}{c^2} +
    \partial_\mu p~g^{\mu\nu} = \\
    %
    \partial_\mu \rho \frac{u^\mu u^\nu}{c^2} +
    \partial_\mu p \frac{u^\mu u^\nu}{c^2} +
    \frac{1}{c^2}(\rho + p)\partial_\mu u^\mu u^\nu +
    \frac{1}{c^2}(\rho + p) u^\mu \partial_\mu u^\nu +
    \partial_\mu p~g^{\mu\nu} = \\
    %
    \frac{1}{c^2}\left(
    u^\nu u^\mu \partial_\mu \rho  +
    u^\nu u^\mu \partial_\mu p  +
    (\rho + p)\partial_\mu u^\mu u^\nu +
    (\rho + p) u^\mu \partial_\mu u^\nu +
    c^2 \partial_\mu p~g^{\mu\nu}\right)
\end{multline}
%
Now we define $\frac{\dd}{\dd \tau} \equiv u^\mu \partial_\mu$ and
plug it into equation
%
\begin{multline}
    \frac{1}{c^2}\left(
    u^\nu \frac{\dd \rho}{\dd \tau}  +
    u^\nu u^\mu \partial_\mu p  +
    u^\nu (\rho + p)\partial_\mu u^\mu  +
    (\rho + p) \frac{\dd u^\nu}{\dd \tau} +
    c^2 \partial_\mu p~g^{\mu\nu}\right) =\\
    %
    \frac{1}{c^2} u^\nu \left(
    \frac{\dd \rho}{\dd \tau}  +
    (\rho + p)\partial_\mu u^\mu \right)  +
    \frac{1}{c^2} (\rho + p) \frac{\dd u^\nu}{\dd \tau} +
    \left(\frac{u^\nu u^\mu}{c^2} + g^{\mu\nu}\right)  \partial_\mu p
\end{multline}
%
We also define $P^{\mu\nu} \equiv \frac{u^\nu u^\mu}{c^2} + g^{\mu\nu}$. Then
condition $\partial_\mu T^{\mu\nu} = 0$ is equivalent to
%
\begin{equation}
    u^\nu \left(
    \frac{\dd \rho}{\dd \tau}  +
    (\rho + p)\partial_\mu u^\mu \right)  +
    (\rho + p) \frac{\dd u^\nu}{\dd \tau} +
    c^2 P^{\mu\nu} \partial_\mu p =0
    \label{eq:ass7_prob1}
\end{equation}
%
Now we will show that $P\indices{^\mu_\nu}$ is projector on the plane orthogonal
to \bs{u}.
%
\begin{equation}
    P\indices{^\mu_\nu} = P\indices{^\mu^\sigma} g_{\sigma\nu} =
    \frac{u^\sigma u^\mu}{c^2} g_{\sigma\nu} + g^{\mu\sigma} g_{\sigma\nu} =
    \frac{u_\nu u^\mu}{c^2} + \delta^\mu_\nu
\end{equation}
%
\begin{enumerate}
    \item Idempotency
          \begin{equation}
              P^2 = P \implies P\indices{^\mu_\rho} P\indices{^\rho_\nu} = P\indices{^\mu_\nu}
          \end{equation}
          \begin{multline}
              \left(\frac{u_\rho u^\mu}{c^2} + \delta^\mu_\rho\right)
              \left(\frac{u_\nu u^\rho}{c^2} + \delta^\rho_\nu\right) =
              \frac{u_\rho u^\rho u^\mu u_\nu }{c^2} +
              \frac{u_\rho u^\mu}{c^2} \delta^\rho_\nu +
              \delta^\mu_\rho \frac{u_\nu u^\rho}{c^2} +
              \delta^\mu_\rho  \delta^\rho_\nu = \\
              -\frac{u^\mu u_\nu }{c^2} +
              \frac{u_\nu u^\mu}{c^2}  +
              \frac{u_\nu u^\mu}{c^2} +
              \delta^\mu_\nu =
              \frac{u_\nu u^\mu}{c^2} +
              \delta^\mu_\nu = P\indices{^\mu_\nu}
          \end{multline}
    \item Gives 0 when act on \bs{u}
          \begin{equation}
              P\bs{u} = 0 \implies P\indices{^\mu_\nu} u^\nu =0
          \end{equation}
          \begin{equation}
              \left(\frac{u_\nu u^\mu}{c^2} + \delta^\mu_\nu\right) u^\nu =
              \frac{u^\nu u_\nu u^\mu}{c^2} + \delta^\mu_\nu u^\nu =
              -u^\mu + u^\mu = 0
          \end{equation}
    \item Gives $\bs{v}$ when act on $\bs{v}\perp\bs{u}$
          \begin{equation}
              P\bs{v} = \bs{v} \implies P\indices{^\mu_\nu} v^\nu = v^\mu
          \end{equation}
          \begin{equation}
              \left(\frac{u_\nu u^\mu}{c^2} + \delta^\mu_\nu\right) v^\nu =
              \frac{\overbrace{v^\nu u_\nu}^{=0} u^\mu}{c^2} + \delta^\mu_\nu v^\nu =
              v^\mu
          \end{equation}
\end{enumerate}
%
So now we can see that \cref{eq:ass7_prob1} consist of two parts which are
orthogonal (so independent) \footnote{4-acceleration is also orthogonal to
    4-velocity}.
%
\begin{subequations}
    \begin{tcolorbox}[ams align]
        \frac{\dd \rho}{\dd \tau}  +
        (\rho + p)\partial_\mu u^\mu  & = 0 \\
        (\rho + p) \frac{\dd u^\nu}{\dd \tau} +
        c^2 P^{\mu\nu} \partial_\mu p & = 0
    \end{tcolorbox}
\end{subequations}

\problem

Now we can substitute $\frac{\dd}{\dd \tau} = u^\mu \partial_\mu$ and $u^\mu =
    \gamma(c,v^i)$ we obtain
%
\begin{subequations}
    \begin{align}
        \frac{1}{c} u^0 \partial_t \rho +
        u^i \partial_i \rho  +
        \frac{1}{c} (\rho + p)\partial_t u^0 +
        (\rho + p)\partial_i u^i      & = 0 \\
        (\rho + p) u^\mu \partial_\mu u^\nu +
        u^\nu u^\mu \partial_\mu p +
        c^2 g^{\mu\nu} \partial_\mu p & = 0
    \end{align}
\end{subequations}
%
Let's first investigate first equation
%
\begin{multline}
    \frac{1}{c} u^0 \partial_t \rho +
    u^i \partial_i \rho  +
    \frac{1}{c} (\rho + p)\partial_t u^0 +
    (\rho + p)\partial_i u^i  = \\
    %
    \gamma \left(\partial_t \rho +
    v^i \nabla_i \rho  +
    (\rho + p)\nabla_i v^i\right)  =
    %
    \gamma \left(\partial_t \rho +
    \nabla_i \left(v^i  \rho\right)  +
    p \nabla_i v^i\right) =
    %
    \gamma \left(\partial_t \rho +
    \bs{\nabla} \left(\bs{v}  \rho\right)  +
    p \bs{\nabla} \bs{v}\right) = \\
    %
    /\text{neglecting terms with $p$}/ \hspace{1cm}
    %
    \gamma \left(\partial_t \rho +
    \bs{\nabla} \left(\bs{v}  \rho\right)\right) = 0
\end{multline}
%
which gives
%
\begin{equation}
    \boxed{\partial_t \rho +
        \bs{\nabla} \left(\bs{v}  \rho\right) = 0}
\end{equation}
%
Now second equation
%
\begin{multline}
    (\rho + p) u^\mu \partial_\mu u^\nu +
    u^\nu u^\mu \partial_\mu p +
    c^2 g^{\mu\nu} \partial_\mu p  =  \\
    %
    \frac{1}{c} (\rho + p) u^0 \partial_t u^\nu +
    (\rho + p) u^i \partial_i u^\nu +
    \frac{1}{c} u^\nu u^0 \partial_t p +
    u^\nu u^i \partial_i p +
    c g^{0\nu} \partial_t p +
    c^2 g^{i\nu} \partial_i p =  \\
    %
    \gamma (\rho + p) \partial_t u^\nu +
    \gamma (\rho + p) v^i \nabla_i u^\nu +
    \gamma u^\nu \partial_t p +
    \gamma u^\nu v^i \nabla_i p -
    \delta^\nu_0 c \partial_t p +
    \delta^\nu_i c^2 \nabla_i p
\end{multline}
%
I split this equation into two cases: $\nu = 0$ and $\nu = i$
\begin{itemize}
    \item[$\nu=0$]
          \begin{equation}
              \gamma (\rho + p) \partial_t u^0 +
              \gamma (\rho + p) v^i \nabla_i u^0 +
              \gamma u^0 \partial_t p +
              \gamma u^0 v^i \nabla_i p -
              \delta^0_0 c \partial_t p = \\
              %
              \gamma^2 c \partial_t p +
              \gamma^2 c v^i \nabla_i p -
              c \partial_t p
          \end{equation}
          Plugging in $\gamma = 1$ gives
          \begin{equation}
              \boxed{\bs{v}\bs{\nabla}p = 0}
          \end{equation}
    \item[$\nu=i$]
          \begin{multline}
              \gamma (\rho + p) \partial_t u^i +
              \gamma (\rho + p) v^i \nabla_i u^i +
              \gamma u^i \partial_t p +
              \gamma u^i v^i \nabla_i p +
              c^2 \nabla_i p    = \\
              %
              \gamma (\rho + p) \partial_t v^j +
              \gamma (\rho + p) v^i \nabla_i v^j +
              \gamma v^j \partial_t p +
              \gamma v^j v^i \nabla_i p +
              c^2 \nabla_j p    = \\
              %
              \gamma \left(p \partial_t v^j + v^j \partial_t p\right) +
              \gamma \left(p v^i \nabla_i v^j + v^j v^i \nabla_i p\right) +
              \gamma \rho v^i \nabla_i v^j +
              \gamma \rho \partial_t v^j +
              c^2 \nabla_j p    = \\
              %
              \gamma \partial_t\left(p v^j\right) +
              \gamma v^i \nabla_i \left(p  v^j\right) +
              \gamma \rho \left(\partial_t + v^i \nabla_i\right)v^j +
              c^2 \nabla_j p    = \\
              %
              \gamma \left(\partial_t +  \bs{v} \bs{\nabla}\right) \left(p \bs{v}\right) +
              \gamma \rho \left(\partial_t + \bs{v} \bs{\nabla}\right)\bs{v} +
              c^2 \bs{\nabla} p    = 0
          \end{multline}
          Dividing both sides by $c^2$, neglecting terms $\frac{p}{c^2}$ and
          plugging $\gamma = 1$ gives us
          %
          \begin{equation}
              \boxed{\frac{\rho}{c^2} \left(\partial_t + \bs{v} \bs{\nabla}\right)\bs{v} +
                  \bs{\nabla} p    = 0}
          \end{equation}
\end{itemize}
%
Final result is
%
\begin{subequations}
    \begin{tcolorbox}[ams align]
        \partial_t \rho + \bs{\nabla}\left(\rho \bs{v}\right) & = 0              \\
        \frac{\rho}{c^2} \left(\partial_t  +
        \bs{v} \bs{\nabla} \right) \bs{v}                     & = -\bs{\nabla} p
    \end{tcolorbox}
\end{subequations}
%
The first equation is \textbf{continuity equations}. It states that change in
energy density in some volume is precisely the flow of the energy desity out (or
in) this volume. So overall energy is conserved.

\smallskip

The second equation is \textbf{Navier--Stockes equation} which describes motion
of viscous fluid.
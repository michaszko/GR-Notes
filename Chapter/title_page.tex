\thispagestyle{empty}

\begin{center}
	\scshape

	\vspace*{2cm}

	\hrule

	\vspace{.7cm}

	{\LARGE{General relativity -- solutions }}

	\vspace{.7cm}

	\hrule

	\vspace{2cm}

	{\large{\textit{Author}:}}\\
	{\large{Michał Siemaszko}}

	\vspace*{2cm}

	\begin{minipage}{0.8\linewidth}
		\begin{flushleft}
			\normalfont
			Those notes were created during course on General relativity in
			summer semester 2019/2020, so during Coronavirus pandemic. We had
			tutorials online and we had to solve problems anyway. Instead of
			solving them on paper and then scanning I decided to improve my
			\LaTeX~skill with \textit{Tikz} package (for plotting). Here is my
			final result.
		\end{flushleft}
	\end{minipage}

	\vspace*{2cm}

	\begin{figure}[H]
		\centering
		\begin{tikzpicture}[domain=0:20]
			\pgfmathsetmacro{\g}{1.5}
			\pgfmathsetmacro{\gg}{1/\g}
			\begin{axis}
				[
					axis lines  = center,
					xlabel={$\tau$},
					ylabel={$a(\tau)$},
					xmin=0,
					xmax=3.5,
					ymin=0,
					ymax=3.5,
					ticks=none
				]

				\addplot[domain = 0:3.14, samples = 1000,color=blue, thick]
				gnuplot {sin(x)};
				\addplot[domain = 0:3.14, samples = 1000,color=red, thick]
				gnuplot {x};
				\addplot[domain = 0:3.14, samples = 1000,color=green, thick]
				gnuplot {sinh(x)};

				\node[inner sep=2pt,text=red] at (2.5,2) {$k=0$};
				\node[inner sep=2pt,text=green] at (.5,1.5) {$k=-1$};
				\node[inner sep=2pt,text=blue] at (1.5,.5) {$k=1$};
			\end{axis}
		\end{tikzpicture}
		% \caption{Radiation dominated universe}
	\end{figure}

	\vfill

	\vspace{1cm}

	{\large{\today}}

	\upshape

\end{center}

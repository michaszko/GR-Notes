\mytitle{ASSIGMENT 3}

\section*{Problem 1}
Calculate EOM given the Lagrangian
%
\begin{equation}
    \mathcal{L}_{\text{dyn}}(x^\mu,\dot{x}^\mu) =
    \frac{1}{2}g_{\mu\nu}[x^\mu(\lambda)]\dot{x}^\mu\dot{x}^\nu, \quad
    \dot{x}^{\mu} \equiv \frac{\di x^\mu}{\dd\lambda}
\end{equation}
%
I calculate first variation of Lagrangian
%
\begin{multline}
    \delta\int\limits_{\lambda1}^{\lambda2}\mathcal{L}(x^\mu,\dot{x}^\mu)\dd\lambda =
    \int\limits_{\lambda1}^{\lambda2}
    \left(\frac{\partial\mathcal{L}(x^\mu,\dot{x}^\mu)}{\partial x^\sigma}\delta x^\sigma +
    \frac{\partial\mathcal{L}(x^\mu,\dot{x}^\mu)}{\partial \dot{x}^\sigma}\delta \dot{x}^\sigma\right)\dd\lambda = \\
    \int\limits_{\lambda1}^{\lambda2}
    \left(\frac{1}{2}\dot{x}^\mu\dot{x}^\nu \partial_\sigma g_{\mu\nu} \delta x^\sigma +
    \frac{1}{2}g_{\mu\nu}(\delta_{\mu\sigma} \dot{x}^\nu + \delta_{\nu\sigma} \dot{x}^\mu)\delta \dot{x}^\sigma\right)\dd\lambda
    \label{eq:ass3_prob1}
\end{multline}
%
Let's take a look at second part of the integral:
%
\begin{multline}
    \frac{1}{2}\int\limits_{\lambda1}^{\lambda2}
    \left\{g_{\mu\nu}(\delta_{\mu\sigma}
    \dot{x}^\nu + \delta_{\nu\sigma} \dot{x}^\mu)\delta \dot{x}^\sigma\right\}\dd\lambda =
    %
    \frac{1}{2}\int\limits_{\lambda1}^{\lambda2}
    \left\{g_{\sigma\nu}\dot{x}^\nu + g_{\mu\sigma}\dot{x}^\mu \right\}
    \delta\dot{x}^\sigma\dd\lambda = \\
    %
    \frac{1}{2}\int\limits_{\lambda1}^{\lambda2}
    \frac{\partial}{\partial\lambda}\left(\left\{
    g_{\sigma\nu}\dot{x}^\nu + g_{\mu\sigma}\dot{x}^\mu
    \right\}\delta x^\sigma\right)\dd\lambda -
    \frac{1}{2}\int\limits_{\lambda1}^{\lambda2}
    \frac{\partial}{\partial\lambda}\left\{
    g_{\sigma\nu}\dot{x}^\nu + g_{\mu\sigma}\dot{x}^\mu
    \right\}\delta x^\sigma\dd\lambda= \\
    %
    \frac{1}{2}\underbrace{\left\{g_{\sigma\nu}\dot{x}^\nu + g_{\mu\sigma}\dot{x}^\mu\right\} \delta x^\sigma \Bigg|_{\lambda1}^{\lambda2}}_{=0} -
    \frac{1}{2}\int\limits_{\lambda1}^{\lambda2}
    \left\{
    \partial_{\mu}g_{\sigma\nu}\dot{x}^\mu\dot{x}^\nu + g_{\sigma\nu}\ddot{x}^\nu +
    \partial_{\nu}g_{\sigma\mu}\dot{x}^\nu\dot{x}^\mu + g_{\sigma\mu}\ddot{x}^\mu
    \right\}\delta x^\sigma\dd\lambda \\
    %
\end{multline}
%
Plugging result back into \autoref{eq:ass3_prob1} yields
%
\begin{equation}
    \delta\int\limits_{\lambda1}^{\lambda2}\mathcal{L}(x^\mu,\dot{x}^\mu)\dd\lambda =
    \frac{1}{2}\int\limits_{\lambda1}^{\lambda2}
    \left(\partial_\sigma g_{\mu\nu}\dot{x}^\mu\dot{x}^\nu -
    \partial_{\mu}g_{\sigma\nu}\dot{x}^\mu\dot{x}^\nu - g_{\sigma\nu}\ddot{x}^\nu -
    \partial_{\nu}g_{\sigma\mu}\dot{x}^\nu\dot{x}^\mu - g_{\sigma\mu}\ddot{x}^\mu\right)\delta x^\sigma \dd\lambda
\end{equation}
%
We want
%
\begin{equation}
   \delta\int\limits_{\lambda1}^{\lambda2}\mathcal{L}(x^\mu,\dot{x}^\mu)\dd\lambda = 0
\end{equation}
%
but since $\delta x^\sigma$ can be arbitrary the rest has to be equal $0$, namely
%
\begin{equation}
    \partial_\sigma g_{\mu\nu}\dot{x}^\mu\dot{x}^\nu -
    \partial_{\mu}g_{\sigma\nu}\dot{x}^\mu\dot{x}^\nu - g_{\sigma\nu}\ddot{x}^\nu -
    \partial_{\nu}g_{\sigma\mu}\dot{x}^\nu\dot{x}^\mu - g_{\sigma\mu}\ddot{x}^\mu = 0
\end{equation}
%
or after rearranging elements
%
\begin{equation}
    \boxed{2 g_{\sigma\mu}\ddot{x}^\mu + \left(\partial_{\nu}g_{\sigma\mu} +
        \partial_{\mu}g_{\sigma\nu} - \partial_\sigma g_{\mu\nu}\right)\dot{x}^\mu\dot{x}^\nu = 0}
\end{equation}

\section*{Problem 2}

Calculate the EOM given Lagrangian:

\begin{equation}
    \mathcal{L}_{\text{geo}}(x^\mu,\dot{x}^\mu) =
    \sqrt{-g_{\mu\nu}[x^\mu(\lambda)]\dot{x}^\mu\dot{x}^\nu}, \quad
    \dot{x}^{\mu} \equiv \frac{\di x^\mu}{\dd\lambda}
\end{equation}
%
First we calculate variation of Lagrangian
%
\begin{multline}
    \delta\int\limits_{\lambda1}^{\lambda2}\mathcal{L}(x^\mu,\dot{x}^\mu)\dd\lambda =
    %
    \int\limits_{\lambda1}^{\lambda2}
    \left(\frac{\partial\mathcal{L}(x^\mu,\dot{x}^\mu)}{\partial x^\sigma}\delta x^\sigma +
    \frac{\partial\mathcal{L}(x^\mu,\dot{x}^\mu)}{\partial \dot{x}^\sigma}\delta \dot{x}^\sigma\right)\dd\lambda = \\
    %
    -\int\limits_{\lambda1}^{\lambda2}
    \left(\frac{\dot{x}^\mu\dot{x}^\nu \partial_\sigma g_{\mu\nu}}{2\sqrt{-g_{\mu\nu}[x^\mu(\lambda)]\dot{x}^\mu\dot{x}^\nu}}\delta x^\sigma +
    \frac{g_{\mu\nu}\dot{x}^\mu\delta_{\nu\sigma}}{2\sqrt{-g_{\mu\nu}[x^\mu(\lambda)]\dot{x}^\mu\dot{x}^\nu}} \dot{x}^\sigma +
    \frac{g_{\mu\nu}\delta_{\mu\sigma}\dot{x}^\nu}{2\sqrt{-g_{\mu\nu}[x^\mu(\lambda)]\dot{x}^\mu\dot{x}^\nu}} \dot{x}^\sigma\right)\dd\lambda = \\
    %
    -\int\limits_{\lambda1}^{\lambda2}\frac{1}{2\sqrt{-g_{\mu\nu}[x^\mu(\lambda)]\dot{x}^\mu\dot{x}^\nu}}
    \left(\dot{x}^\mu\dot{x}^\nu \partial_\sigma g_{\mu\nu} \delta x^\sigma +
    g_{\mu\sigma}\dot{x}^\mu \delta \dot{x}^\sigma +
    g_{\sigma\nu}\dot{x}^\nu \delta \dot{x}^\sigma \right)\dd\lambda
    \label{eq:ass3_prob2_1}
\end{multline}
%
I change variable of differentiating and integration from $\dd\lambda$ to
$\dd\tau = \sqrt{-g_{\mu\nu}[x^\mu(\lambda)]\dot{x}^\mu\dot{x}^\nu}\dd\lambda$.
Derivatives changing as following
%
\begin{equation}
    \dot{x}^{\mu} \equiv \frac{\di x^\mu}{\dd\lambda} =
    \frac{\di x^\mu}{\dd\tau} \frac{\dd\tau}{\dd\lambda} =
    \sqrt{-g_{\mu\nu}[x^\mu(\lambda)]\dot{x}^\mu\dot{x}^\nu}~\frac{\di x^\mu}{\dd\tau}
\end{equation}
%
and integral as following
%
\begin{equation}
    \int\dd\lambda =
    \int\frac{\dd\tau}{\sqrt{-g_{\mu\nu}[x^\mu(\lambda)]\dot{x}^\mu\dot{x}^\nu}}
\end{equation}
%
After plugging in those transformations into \autoref{eq:ass3_prob2_1} we obtain
%
\begin{equation}
    -\frac{1}{2}\int\limits_{\lambda1}^{\lambda2}
    \left(\frac{\dd x^\mu}{\dd\tau}\frac{\dd x^\nu}{\dd\tau} \partial_\sigma g_{\mu\nu} \delta x^\sigma +
    \left\{g_{\mu\sigma}\frac{\dd x^\mu}{\dd\tau} +
    g_{\sigma\nu}\frac{\dd x^\nu}{\dd\tau}\right\} \delta \frac{\dd x^\sigma}{\dd\tau} \right)\dd\tau
    \label{eq:ass3_prob2_2}
\end{equation}
%
Now let's look at the second part of this integral and transform it (using
Leibniz rule)
%
\begin{multline}
    \left\{g_{\mu\sigma}\frac{\dd x^\mu}{\dd\tau} +
    g_{\sigma\nu}\frac{\dd x^\nu}{\dd\tau}\right\} \delta \frac{\dd x^\sigma}{\dd\tau} =
    \frac{\dd}{\dd \tau} \left(\left\{g_{\mu\sigma}\frac{\dd x^\mu}{\dd\tau} +
    g_{\sigma\nu}\frac{\dd x^\nu}{\dd\tau}\right\}\delta x^\sigma\right) -
    \frac{\dd}{\dd \tau} \left\{g_{\mu\sigma}\frac{\dd x^\mu}{\dd\tau} +
    g_{\sigma\nu}\frac{\dd x^\nu}{\dd\tau}\right\}\delta x^\sigma = \\
    \frac{\dd}{\dd \tau} \left(\left\{g_{\mu\sigma}\frac{\dd x^\mu}{\dd\tau} +
    g_{\sigma\nu}\frac{\dd x^\nu}{\dd\tau}\right\}\delta x^\sigma\right) -
    \left\{
    \frac{\dd g_{\mu\sigma}}{\dd x^\nu}\frac{\dd x^\nu}{\dd\tau}\frac{\dd x^\mu}{\dd\tau} +
    g_{\mu\sigma}\frac{\dd^2 x^\mu}{\dd \tau^2} +
    \frac{\dd g_{\sigma\nu}}{\dd x^\mu}\frac{\dd x^\mu}{\dd\tau}\frac{\dd x^\nu}{\dd\tau}+
    g_{\sigma\nu}\frac{\dd^2 x^\nu}{\dd \tau^2} \right\}\delta x^\sigma = \\
    \frac{\dd}{\dd \tau} \left(\left\{g_{\mu\sigma}\frac{\dd x^\mu}{\dd\tau} +
    g_{\sigma\nu}\frac{\dd x^\nu}{\dd\tau}\right\}\delta x^\sigma\right) -
    \left[g_{\mu\sigma}\frac{\dd^2 x^\mu}{\dd \tau^2} +
        g_{\sigma\nu}\frac{\dd^2 x^\nu}{\dd \tau^2} +
        \frac{\dd x^\mu}{\dd\tau}\frac{\dd x^\nu}{\dd\tau}
        \left\{\partial_\nu g_{\mu\sigma} + \partial_\mu g_{\sigma\nu}\right\}\right]\delta x^\sigma
\end{multline}
%
After plugging it into \autoref{eq:ass3_prob2_2} we obtain
%
\begin{multline}
    -\frac{1}{2}\int\limits_{\lambda1}^{\lambda2}
    \Bigg(\frac{\dd x^\mu}{\dd\tau}\frac{\dd x^\nu}{\dd\tau} \partial_\sigma g_{\mu\nu} \delta x^\sigma +
    \frac{\dd}{\dd \tau} \left[\left\{g_{\mu\sigma}\frac{\dd x^\mu}{\dd\tau} +
        g_{\sigma\nu}\frac{\dd x^\nu}{\dd\tau}\right\}\delta x^\sigma\right] - \\
    \left[g_{\mu\sigma}\frac{\dd^2 x^\mu}{\dd \tau^2} +
        g_{\sigma\nu}\frac{\dd^2 x^\nu}{\dd \tau^2} +
        \frac{\dd x^\mu}{\dd\tau}\frac{\dd x^\nu}{\dd\tau}
        \left\{\partial_\nu g_{\mu\sigma} + \partial_\mu g_{\sigma\nu}\right\}\right]\delta x^\sigma\Bigg)\dd\tau = \\
    -\frac{1}{2}\int\limits_{\lambda1}^{\lambda2}
    \left[
        -2 g_{\sigma\nu}\frac{\dd^2 x^\nu}{\dd \tau^2} +
        \frac{\dd x^\mu}{\dd\tau}\frac{\dd x^\nu}{\dd\tau}
        \left(
        \partial_\sigma g_{\mu\nu}  -
        \partial_\nu g_{\mu\sigma} -
        \partial_\mu g_{\sigma\nu}\right)\right]\delta x^\sigma \dd\tau-
    \underbrace{\left\{g_{\mu\sigma}\frac{\dd x^\mu}{\dd\tau} +
    g_{\sigma\nu}\frac{\dd x^\nu}{\dd\tau}\right\}\delta x^\sigma \Bigg|_{\lambda1}^{\lambda2}}_{=0} = \\
    -\frac{1}{2}\int\limits_{\lambda1}^{\lambda2}
    \left[
        -2 g_{\sigma\nu}\frac{\dd^2 x^\nu}{\dd \tau^2} +
        \frac{\dd x^\mu}{\dd\tau}\frac{\dd x^\nu}{\dd\tau}
        \left(
        \partial_\sigma g_{\mu\nu}  -
        \partial_\nu g_{\mu\sigma} -
        \partial_\mu g_{\sigma\nu}\right)\right]\delta x^\sigma \dd\tau
\end{multline}
%
But this variation has to be equal zero no matter what the value of $\delta
    x^\sigma$ is. Namely
%
\begin{equation}
    -g_{\sigma\nu}\frac{\dd^2 x^\nu}{\dd \tau^2} +
    \frac{1}{2}\frac{\dd x^\mu}{\dd\tau}\frac{\dd x^\nu}{\dd\tau}
    \left(
    \partial_\sigma g_{\mu\nu}  -
    \partial_\nu g_{\mu\sigma} -
    \partial_\mu g_{\sigma\nu}\right) = 0
\end{equation}
%
Changing sign
%
\begin{equation}
    \boxed{g_{\sigma\nu}\frac{\dd^2 x^\nu}{\dd \tau^2} +
        \frac{1}{2}\frac{\dd x^\mu}{\dd\tau}\frac{\dd x^\nu}{\dd\tau}
        \left(
        \partial_\nu g_{\mu\sigma} +
        \partial_\mu g_{\sigma\nu}-
        \partial_\sigma g_{\mu\nu}\right) = 0}
\end{equation}

\section*{Problem 3}
Following metric is given
%
\begin{equation}
    g_{\mu\nu}\dd x^\mu \dd x^\nu =
    -\left[1+2\frac{\phi(\bs{x})}{c^2}\right] \dd(ct)^2 +
    (\dd x^1)^2 + (\dd x^2)^2 + (\dd x^3)^2, \quad \bs{x} \equiv (x^1,x^2,x^3)
\end{equation}
%
Dividing both sides by $\dd t^2$ yields
%
\begin{multline}
    g_{\mu\nu}\frac{\dd x^\mu}{\dd t} \frac{\dd x^\nu}{\dd t} =
    -\left[1+2\frac{\phi(\bs{x})}{c^2}\right] \frac{\dd(ct)^2}{\dd t^2} +
    \frac{(\dd x^1)^2}{\dd t^2} + \frac{(\dd x^2)^2}{\dd t^2} + \frac{(\dd x^3)^2}{\dd t^2} = \\
    -\left[1+2\frac{\phi(\bs{x})}{c^2}\right] c^2 +
    \left(\frac{\dd x^1}{\dd t}\right)^2 +
    \left(\frac{\dd x^2}{\dd t}\right)^2 +
    \left(\frac{\dd x^3}{\dd t}\right)^2 =
    -\left[1+2\frac{\phi(\bs{x})}{c^2}\right] c^2 +
    \bs{v}\cdot\bs{v} = \\
    -c^2 \left[1+2\frac{\phi(\bs{x})}{c^2} -
        \frac{\bs{v}^2}{c^2}\right]
\end{multline}
%
Substituting this into lagrangian
%
\begin{equation}
    \mathcal{L} = -mc\sqrt{-g_{\mu\nu}\frac{\dd x^\mu}{\dd t}\frac{\dd x^\nu}{\dd t}}
\end{equation}
%
gives us
%
\begin{equation}
    \boxed{\mathcal{L} = -mc \sqrt{c^2 \left[1+2\frac{\phi(\bs{x})}{c^2} -
            \frac{\bs{v}^2}{c^2}\right]} =
    -mc^2\sqrt{1+2\frac{\phi(\bs{x})}{c^2} -
        \frac{\bs{v}^2}{c^2}}}
\end{equation}
%
We can now assume that both $\frac{\phi(\bs{x})}{c^2}$ and
$\frac{\bs{v}^2}{c^2}$ are small. We can taylor expand square root ($\sqrt{1+x}
\simeq 1 + \frac{1}{2x}$) and leave only linear terms:
%
\begin{equation}
    \mathcal{L} = -mc^2 \left(1 + \frac{\phi(\bs{x})}{c^2} - \frac{\bs{v}^2}{2c^2}\right) = 
    -mc^2 - m\phi(\bs{x}) + m\frac{\bs{v}^2}{2}
\end{equation}
%
But adding constants (in this case $mc^2$) to Lagrangian doesn't change equation
of motions, so effective Lagrangian can be written as 
%
\begin{equation}
    \boxed{\mathcal{L} = \frac{m\bs{v}^2}{2} - m\phi(\bs{x})}
\end{equation}
%
which is exactly the Lagrangian for classical mechanics, which leads to Newton's
law of motion, namely 
%
\begin{equation}
    \frac{\dd^2 \bs{x}}{\dd t^2} = - \nabla\phi(\bs{x})
\end{equation}
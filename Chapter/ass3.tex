\mytitle{assigment 3}

\section*{Problem 1}

\begin{equation}
    \mathcal{L}(x^\mu,\dot{x}^\mu) =
    \frac{1}{2}g_{\mu\nu}[x^\mu(\lambda)]\dot{x}^\mu\dot{x}^\nu, \quad
    \dot{x}^{\mu} \equiv \frac{\di x^\mu}{\dd\lambda}
\end{equation}

\begin{multline}
    \delta\int\limits_{\lambda1}^{\lambda2}\mathcal{L}(x^\mu,\dot{x}^\mu)\dd\lambda =
    \int\limits_{\lambda1}^{\lambda2}
    \left(\frac{\partial\mathcal{L}(x^\mu,\dot{x}^\mu)}{\partial x^\sigma}\delta x^\sigma +
    \frac{\partial\mathcal{L}(x^\mu,\dot{x}^\mu)}{\partial \dot{x}^\sigma}\delta \dot{x}^\sigma\right)\dd\lambda = \\
    \int\limits_{\lambda1}^{\lambda2}
    \left(\frac{1}{2}\dot{x}^\mu\dot{x}^\nu \partial_\sigma g_{\mu\nu} \delta x^\sigma +
    \frac{1}{2}g_{\mu\nu}(\delta_{\mu\sigma} \dot{x}^\nu + \delta_{\nu\sigma} \dot{x}^\mu)\delta \dot{x}^\sigma\right)\dd\lambda
\end{multline}
%
Let's take a look at second part of the integral:
%
\begin{multline}
    \frac{1}{2}\int\limits_{\lambda1}^{\lambda2}
    \left\{g_{\mu\nu}(\delta_{\mu\sigma}
    \dot{x}^\nu + \delta_{\nu\sigma} \dot{x}^\mu)\delta \dot{x}^\sigma\right\}\dd\lambda =
    %
    \frac{1}{2}\int\limits_{\lambda1}^{\lambda2}
    \left\{g_{\sigma\nu}\dot{x}^\nu + g_{\mu\sigma}\dot{x}^\mu \right\}
    \delta\dot{x}^\sigma\dd\lambda = \\
    %
    \frac{1}{2}\int\limits_{\lambda1}^{\lambda2}
    \frac{\partial}{\partial\lambda}\left(\left\{
    g_{\sigma\nu}\dot{x}^\nu + g_{\mu\sigma}\dot{x}^\mu
    \right\}\delta x^\sigma\right)\dd\lambda -
    \frac{1}{2}\int\limits_{\lambda1}^{\lambda2}
    \frac{\partial}{\partial\lambda}\left\{
    g_{\sigma\nu}\dot{x}^\nu + g_{\mu\sigma}\dot{x}^\mu
    \right\}\delta x^\sigma\dd\lambda= \\
    %
    \frac{1}{2}\underbrace{\left\{g_{\sigma\nu}\dot{x}^\nu + g_{\mu\sigma}\dot{x}^\mu\right\} \delta x^\sigma \Bigg|_{\lambda1}^{\lambda2}}_{=0} -
    \frac{1}{2}\int\limits_{\lambda1}^{\lambda2}
    \left\{
    \partial_{\mu}g_{\sigma\nu}\dot{x}^\mu\dot{x}^\nu + g_{\sigma\nu}\ddot{x}^\nu +
    \partial_{\nu}g_{\sigma\mu}\dot{x}^\nu\dot{x}^\mu + g_{\sigma\mu}\ddot{x}^\mu
    \right\}\delta x^\sigma\dd\lambda \\
    %
\end{multline}
\begin{equation}
    \delta\int\limits_{\lambda1}^{\lambda2}\mathcal{L}(x^\mu,\dot{x}^\mu)\dd\lambda =
    \frac{1}{2}\int\limits_{\lambda1}^{\lambda2}
    \left(\partial_\sigma g_{\mu\nu}\dot{x}^\mu\dot{x}^\nu -
    \partial_{\mu}g_{\sigma\nu}\dot{x}^\mu\dot{x}^\nu - g_{\sigma\nu}\ddot{x}^\nu -
    \partial_{\nu}g_{\sigma\mu}\dot{x}^\nu\dot{x}^\mu - g_{\sigma\mu}\ddot{x}^\mu\right)\delta x^\sigma \dd\lambda
\end{equation}
%
We want
%
\begin{equation}
    \delta\int\limits_{\lambda1}^{\lambda2}\mathcal{L}(x^\mu,\dot{x}^\mu)\dd\lambda = 0
\end{equation}
%
but since $\delta x^\sigma$ can be arbitrary the rest has to be equal $0$, namely
%
\begin{equation}
    \partial_\sigma g_{\mu\nu}\dot{x}^\mu\dot{x}^\nu -
    \partial_{\mu}g_{\sigma\nu}\dot{x}^\mu\dot{x}^\nu - g_{\sigma\nu}\ddot{x}^\nu -
    \partial_{\nu}g_{\sigma\mu}\dot{x}^\nu\dot{x}^\mu - g_{\sigma\mu}\ddot{x}^\mu = 0
\end{equation}
%
or after rearranging elements
%
\begin{equation}
    \boxed{2 g_{\sigma\mu}\ddot{x}^\mu + \left(\partial_{\nu}g_{\sigma\mu} + 
    \partial_{\mu}g_{\sigma\nu} - \partial_\sigma g_{\mu\nu}\right)\dot{x}^\mu\dot{x}^\nu = 0}
\end{equation}

% \mytitle{ASSIGMENT 6}

\chapter{}

\problem

Show that
%
\begin{equation}
    R(\bs{u},\bs{v})\bs{z} =
    \nabla_{\bs{u}} \nabla_{\bs{v}} \bs{z} -
    \nabla_{\bs{v}} \nabla_{\bs{u}} \bs{z} -
    \nabla_{[\bs{u},\bs{v}]} \bs{z}
\end{equation}
%
satisfies
%
\begin{equation}
    R(\bs{u},\bs{v})(f\bs{z}) = f R(\bs{u},\bs{v})\bs{z}
\end{equation}
%
\begin{multline}
    R(\bs{u},\bs{v})(f\bs{z}) =
    \nabla_{\bs{u}} \left(\nabla_{\bs{v}} (f\bs{z})\right) -
    \nabla_{\bs{v}} \left(\nabla_{\bs{u}} (f\bs{z})\right) -
    \nabla_{[\bs{u},\bs{v}]} (f\bs{z}) = \\
    %
    \nabla_{\bs{u}} \left(\nabla_{\bs{v}}f~\bs{z} + f \nabla_{\bs{v}}\bs{z}\right) -
    \nabla_{\bs{v}} \left(\nabla_{\bs{u}}f~\bs{z} + f \nabla_{\bs{u}}\bs{z}\right) -
    \nabla_{[\bs{u},\bs{v}]}f~\bs{z} - f~\nabla_{[\bs{u},\bs{v}]}\bs{z} = \\
    %
    \nabla_{\bs{u}} \left(\bs{v}(f)~\bs{z} + f \nabla_{\bs{v}}\bs{z}\right) -
    \nabla_{\bs{v}} \left(\bs{u}(f)~\bs{z} + f \nabla_{\bs{u}}\bs{z}\right) -
    [\bs{u},\bs{v}](f)~\bs{z} - f~\nabla_{[\bs{u},\bs{v}]}\bs{z} = \\
    %
    \nabla_{\bs{u}}\bs{v}(f)~\bs{z} + \bs{v}(f)\nabla_{\bs{u}}\bs{z} +
    \nabla_{\bs{u}}f \nabla_{\bs{v}}\bs{z} + f \nabla_{\bs{u}}\nabla_{\bs{v}}\bs{z} - \\
    \nabla_{\bs{v}}\bs{u}(f)~\bs{z} - \bs{u}(f)\nabla_{\bs{v}}\bs{z} -
    \nabla_{\bs{v}}f \nabla_{\bs{u}}\bs{z} - f \nabla_{\bs{v}}\nabla_{\bs{u}}\bs{z} - \\
    [\bs{u},\bs{v}](f)~\bs{z} - f~\nabla_{[\bs{u},\bs{v}]}\bs{z} = \\
    %
    \mathunderline{red}{\bs{u}(\bs{v}(f))~\bs{z}} +
    \mathunderline{blue}{\bs{v}(f)\nabla_{\bs{u}}\bs{z}} +
    \mathunderline{green}{\bs{u}(f) \nabla_{\bs{v}}\bs{z}} +
    f \nabla_{\bs{u}}\nabla_{\bs{v}}\bs{z} - \\
    \mathunderline{red}{\bs{v}(\bs{u}(f))~\bs{z}} -
    \mathunderline{green}{\bs{u}(f)\nabla_{\bs{v}}\bs{z}} -
    \mathunderline{blue}{\bs{v}(f) \nabla_{\bs{u}}\bs{z}} -
    f \nabla_{\bs{v}}\nabla_{\bs{u}}\bs{z} - \\
    \mathunderline{red}{[\bs{u},\bs{v}](f)~\bs{z}} - f~\nabla_{[\bs{u},\bs{v}]}\bs{z} = \\
    f\left(\nabla_{\bs{u}} \nabla_{\bs{v}} \bs{z} -
    \nabla_{\bs{v}} \nabla_{\bs{u}} \bs{z} -
    \nabla_{[\bs{u},\bs{v}]} \bs{z}\right)
\end{multline}

\problem

Show that
%
\begin{equation}
    \delta \left(\ln \det \bs{M} \right) = \text{Tr}\left(\bs{M}^{-1}\delta \bs{M}\right)
    \label{eq:ass6_sol2}
\end{equation}
%
using relation
%
\begin{equation}
    \det(\bs{1}+\epsilon \bs{A}) = 
    1 + \epsilon\text{Tr}\bs{A} + \mathcal{O}(\epsilon^2)
\end{equation}
%
Doing straight forward calculations 
%
\begin{multline}
    \delta(\ln \det \bs{M}) = 
    \lim_{\epsilon\to0}\frac{\ln \det(\bs{M}+\epsilon\delta\bs{M}) - \ln \det \bs{M}}{\epsilon} =
    \lim_{\epsilon\to0}\frac{\ln \frac{\det(\bs{M}+\epsilon\delta\bs{M})}{\det \bs{M}}}{\epsilon} =
    \lim_{\epsilon\to0}\frac{\ln \det (\bs{1}+ \epsilon\delta\bs{M}\bs{M}^{-1})}{\epsilon} = \\
    \lim_{\epsilon\to0}\frac{\ln \left\{1 + \text{Tr} (\epsilon\delta\bs{M}\bs{M}^{-1})\right\}}{\epsilon} =
    /\text{I use Taylor expansion for } \ln(1+x) \simeq x/ \rightarrow \\
    \lim_{\epsilon\to0}\frac{\text{Tr}(\epsilon\delta\bs{M}\bs{M}^{-1})}{\epsilon} = 
    \text{Tr} (\delta\bs{M}\bs{M}^{-1})= 
    \text{Tr} (\bs{M}^{-1}\delta\bs{M})
\end{multline}
%%%%%%%%%%%%%%%%%%%%%
% Old way
%%%%%%%%%%%%%%%%%%%%%
% First we notice that
% %
% \begin{equation}
%     \delta \left(\ln \det \bs{M} \right) = 
%     \frac{\delta \det \bs{M}}{\det \bs{M}} 
%     \label{eq:ass6_prob2}
% \end{equation}
% %
% Now using following relation
% %
% \begin{equation}
%     \det(\bs{1}+\epsilon \bs{A}) = 
%     1 + \epsilon\text{Tr}\bs{A} + \mathcal{O}(\epsilon^2)
% \end{equation}
% %
% We can calculate the expression we will use later
% %
% \begin{equation}
%     \det(\bs{M}+\epsilon\delta\bs{M}) = 
%     \det(\bs{1}+\epsilon\delta\bs{M}\bs{M^{-1}})\det(\bs{M}) =
%     \det\bs{M} + \epsilon\text{Tr}(\delta\bs{M}\bs{M^{-1}})\det\bs{M}
% \end{equation}
% %
% We can calculate derivative using its definition
% %
% \begin{multline}
%     \delta(\det \bs{M}) = 
%     \lim_{\epsilon\to0}\frac{\det(\bs{M}+\epsilon\delta\bs{M}) - \det \bs{M}}{\epsilon} =\\
%     \lim_{\epsilon\to0}\frac{\det\bs{M} + \epsilon\text{Tr}(\delta\bs{M}\bs{M^{-1}})\det\bs{M} - \det \bs{M}}{\epsilon} =
%     \text{Tr}(\delta\bs{M}\bs{M^{-1}})\det\bs{M}
% \end{multline}
% %
% If we plug this into \cref{eq:ass6_prob2} we obtain final result (because trace
% is cyclic).

\problem

Show that 
%
\begin{equation}
    \Gamma_{\nu\mu}^\nu = \partial_\mu \ln \sqrt{|\bs{g}|}
\end{equation}
%
We first calculate
%
\begin{multline}
    \Gamma_{\nu\mu}^\nu = \frac{1}{2}g^{\nu\alpha}\left(
        \partial_\mu g_{\nu\alpha} +
        \partial_\nu g_{\mu\alpha} - 
        \partial_\alpha g_{\nu\mu}
    \right)=
    \frac{1}{2}g^{\nu\alpha}\partial_\mu g_{\nu\alpha} +
    \frac{1}{2}g^{\nu\alpha}\partial_\nu g_{\mu\alpha} - 
    \frac{1}{2}g^{\nu\alpha}\partial_\alpha g_{\nu\mu} = \quad (\nu \leftrightarrow \alpha) \\
    %
    \frac{1}{2}g^{\nu\alpha}\partial_\mu g_{\nu\alpha} +
    \frac{1}{2}g^{\alpha\nu}\partial_\alpha g_{\mu\nu} - 
    \frac{1}{2}g^{\nu\alpha}\partial_\alpha g_{\nu\mu} =
    \frac{1}{2}g^{\nu\alpha}\partial_\mu g_{\nu\alpha} =
    \frac{1}{2}\left(\bs{g}^{-1}\right)_{\alpha\nu}\left(\partial_\mu \bs{g}\right)_{\nu\alpha} =
    \frac{1}{2}\text{Tr}\left(\bs{g}^{-1}\partial_\mu\bs{g}\right)
\end{multline}
%
And now we calculate 
%
\begin{equation}
    \partial_\mu \ln \sqrt{|\bs{g}|} = 
    \partial_\mu \ln |\bs{g}|^{\frac{1}{2}} = 
    \frac{1}{2} \partial_\mu \ln |\bs{g}| \stackrel{\cref{eq:ass6_sol2}}{=}  
    \frac{1}{2} \text{Tr}\left(\bs{g}^{-1}\partial_\mu \bs{g}\right)
\end{equation}
%%%%%%%%%%%%%%%
% Old version
%%%%%%%%%%%%%%%
% \begin{equation}
%     \partial_\mu \ln \sqrt{|\bs{g}|} = 
%     \partial_\mu \ln |\sqrt{\bs{g}}|\stackrel{\cref{eq:ass6_sol2}}{=} 
%     \text{Tr}\left(\sqrt{\bs{g}^{-1}}\partial_\mu \sqrt{\bs{g}}\right) =
%     \text{Tr}\left(\bs{g}^{-\frac{1}{2}}\bs{g}^{-\frac{1}{2}}\frac{1}{2}\partial_\mu\bs{g}\right) =
%     \frac{1}{2}\text{Tr}\left(\bs{g}^{-1}\partial_\mu\bs{g}\right)
% \end{equation}
%
So both sides of equation are equal

\problem

Show that
%
\begin{equation}
    \nabla_\mu A^\mu = \frac{1}{\sqrt{|\bs{g}|}}\partial_\mu
    \left(\sqrt{|\bs{g}|}A^\mu\right) 
\end{equation}
%
We start with 
%
\begin{multline}
    \frac{1}{\sqrt{|\bs{g}|}}\partial_\mu
    \left(\sqrt{|\bs{g}|}A^\mu\right) = 
    %
    \frac{1}{\sqrt{|\bs{g}|}}
    \partial_\mu\sqrt{|\bs{g}|}
    A^\mu +
    \frac{1}{\sqrt{|\bs{g}|}}
    \sqrt{|\bs{g}|}
    \partial_\mu A^\mu =
    %
    \partial_\mu\left(\ln\sqrt{|\bs{g}|}\right)
    A^\mu +
    \frac{1}{\sqrt{|\bs{g}|}}
    \sqrt{|\bs{g}|}
    \partial_\mu A^\mu =\\
    %
    \Gamma_{\nu\mu}^\nu A^\mu + 
    \partial_\mu A^\mu
\end{multline}
%
So 
%
\begin{equation}
    \nabla_\mu A^\mu =
    \Gamma_{\nu\mu}^\nu A^\mu + 
    \partial_\mu A^\mu
\end{equation}
%
which is true by definition of covariant derivative

\problem

We start with 
%
\begin{equation}
    \frac{\dd}{\dd\lambda}\left\{
        \bs{g}\left(\partial_{\sigma^\ast}, \bs{v}\right)
    \right\} = 0
\end{equation}
%
Since we stay on geodesic we can write 
%
\begin{equation}
    \frac{\dd}{\dd\lambda}\left\{
        \bs{g}\left(\partial_{\sigma^\ast}, \bs{v}\right)
    \right\} =
    %
    \nabla_{\bs{v}}\left\{
        \bs{g}\left(\partial_{\sigma^\ast}, \bs{v}\right)
    \right\} 
\end{equation}
%
We can now expand
%
\begin{equation}
    \bs{g}\left(\partial_{\sigma^\ast}, \bs{v}\right) =
    \delta^\nu_{\sigma^\ast} g_{\nu \mu}v^\mu
\end{equation}
%
So eventually
%
\begin{multline}
    \nabla_{\bs{v}}\left\{
        \delta^\nu_{\sigma^\ast} g_{\nu \mu}v^\mu
    \right\} =
    %
    g_{\nu \mu}
    \nabla_{\bs{v}}\left\{
        \delta^\nu_{\sigma^\ast} v^\mu
    \right\} =
    %
    g_{\nu \mu} \nabla_{\bs{v}}\delta^\nu_{\sigma^\ast}v^\mu +
    g_{\nu \mu} \delta^\nu_{\sigma^\ast}\nabla_{\bs{v}} v^\mu = \\
    %
    g_{\nu \mu} \Gamma_{\rho \kappa}^{\nu} v^\kappa \delta^\rho_{\sigma^\ast} v^\mu +
    g_{\nu \mu} \delta^\nu_{\sigma^\ast} v^\rho \partial_\rho v^\mu + 
    g_{\nu \mu} \delta^\nu_{\sigma^\ast}\Gamma_{\rho \kappa}^{\mu} v^\rho v^\kappa = \\
    %
    g_{\nu \mu} \Gamma_{\sigma^\ast \kappa}^{\nu} v^\kappa  v^\mu +
    g_{\sigma^\ast \mu}  v^\rho \partial_\rho v^\mu + 
    g_{\sigma^\ast \mu} \Gamma_{\rho \kappa}^{\mu} v^\rho v^\kappa =
\end{multline}
%
Now I calculate 
%
\begin{equation}
    g_{\nu \mu} \Gamma_{\sigma^\ast \kappa}^{\nu} =
    %
    \frac{1}{2} \underbrace{g_{\nu \mu}  g^{\nu\alpha}}_{ = \delta_\mu^\alpha}
    (\partial_\kappa g_{{\sigma^\ast}\alpha} +
    \underbrace{\partial_{\sigma^\ast} g_{\kappa\alpha}}_{ = 0} - 
    \partial_\alpha g_{\kappa{\sigma^\ast}}) = 
    %
    \frac{1}{2} 
    (\partial_\kappa g_{{\sigma^\ast}\mu} - 
    \partial_\mu g_{\kappa{\sigma^\ast}}) 
\end{equation}
%
and 
%
\begin{equation}
    g_{\sigma^\ast \mu} \Gamma_{\rho \kappa}^{\mu} = 
    %
    \frac{1}{2} \underbrace{g_{\sigma^\ast \mu}  g^{\mu\alpha}}_{ = \delta_{\sigma^\ast}^\alpha}
    (\partial_\kappa g_{{\rho}\alpha} +
    \partial_{\rho} g_{\kappa\alpha} - 
    \partial_\alpha g_{\kappa{\rho}}) = 
    %
    \frac{1}{2} 
    (\partial_\kappa g_{{\rho}\sigma^\ast} +
    \partial_{\rho} g_{\kappa\sigma^\ast} - 
    \underbrace{\partial_{\sigma^\ast} g_{\kappa{\rho}}}_{ = 0}) = 
    %
    \frac{1}{2} 
    (\partial_\kappa g_{{\rho}\sigma^\ast} +
    \partial_{\rho} g_{\kappa\sigma^\ast})  
\end{equation}
%
Plugging down everything we have
%
\begin{multline}
    g_{\nu \mu} \Gamma_{\sigma^\ast \kappa}^{\nu} v^\kappa  v^\mu +
    g_{\sigma^\ast \mu}  v^\rho \partial_\rho v^\mu + 
    g_{\sigma^\ast \mu} \Gamma_{\rho \kappa}^{\mu} v^\rho v^\kappa = \\
    %
    \frac{1}{2} \underbrace{(\partial_\kappa g_{{\sigma^\ast}\mu} - 
    \partial_\mu g_{\kappa{\sigma^\ast}}) v^\kappa  v^\mu }_{ = 0} +
    g_{\sigma^\ast \mu}  v^\rho \partial_\rho v^\mu + 
    \frac{1}{2} (\partial_\kappa g_{{\rho}\sigma^\ast} +
    \partial_{\rho} g_{\kappa\sigma^\ast})  v^\rho v^\kappa = \\
    %
    g_{\sigma^\ast \mu}  v^\rho \partial_\rho v^\mu + 
    \partial_\kappa g_{{\rho}\sigma^\ast} v^\rho v^\kappa = 
    %
    v^\rho \partial_\rho \left(g_{\sigma^{\star}\mu}v^\mu\right)
\end{multline}

\chapter

\problem

\subproblem

We have given metric
%
\begin{equation}
    \bs{g} = \dd\theta \otimes \dd\theta + \sin^2\theta~\dd\phi \otimes \dd\phi
\end{equation}
%
Only non-zero elements are:
%
\begin{equation}
    g_{11} = 1 \hspace{2cm} g_{22} = \sin^2\theta
\end{equation}
%
Also rmeber that $g^{\alpha\beta} = g_{\alpha\beta}^{-1}$ Let's calculate
connection for that metric:
%
\begin{equation}
    \Gamma^\mu_{\nu\sigma} =
    \frac{1}{2} g^{\mu\rho} \left(
    \partial_\nu g_{\rho\sigma} +
    \partial_\sigma g_{\rho\nu} -
    \partial_\rho g_{\nu\sigma}\right)
\end{equation}
%
It is easy to see that
%
\begin{equation}
    \partial_\nu g_{\rho\sigma} =
    \delta_{1\nu}\delta_{2\rho}\delta_{2\sigma}~2\sin\theta\cos\theta
    \label{eq:ass5_only_non_zero}
\end{equation}
%
where $\partial_1 \equiv \partial_\theta$ and $\partial_2 \equiv \partial_\phi$.
Substituting \cref{eq:ass5_only_non_zero} into connection yields
%
\begin{multline}
    \Gamma^\mu_{\nu\sigma} =
    \frac{1}{2} g^{\mu\rho} \left(
    \delta_{1\nu}\delta_{2\rho}\delta_{2\sigma}~2\sin\theta\cos\theta +
    \delta_{1\sigma}\delta_{2\rho}\delta_{2\nu}~2\sin\theta\cos\theta-
    \delta_{1\rho}\delta_{2\nu}\delta_{2\sigma}~2\sin\theta\cos\theta\right) =\\
    \sin\theta\cos\theta\left(
    g^{2\mu} \delta_{1\nu}\delta_{2\sigma} +
    g^{2\mu} \delta_{1\sigma}\delta_{2\nu} -
    g^{1\mu}\delta_{2\nu}\delta_{2\sigma}\right)
\end{multline}
%
The only non--zero coefficients
%
\begin{equation}
    \Gamma^1_{22} = -\sin\theta\cos\theta
    \hspace*{2cm}
    \Gamma^2_{12} = \Gamma^2_{21} = \frac{\cos\theta}{\sin\theta}
    \label{eq:ass5_coeff}
\end{equation}

\subproblem

Geodesic equation is given by
%
\begin{equation}
    \frac{\partial^2 x^\sigma}{\partial \lambda^2} +
    \Gamma^\sigma_{\mu\nu}
    \frac{\partial x^\mu}{\partial \lambda}
    \frac{\partial x^\nu}{\partial \lambda} = 0
\end{equation}
%
Writing explicitly
%
\begin{subequations}
    \begin{align}
        \frac{\partial^2 x^1}{\partial \lambda^2} +
        \Gamma^{1}_{\mu\nu}
        \frac{\partial x^\mu}{\partial \lambda}
        \frac{\partial x^\nu}{\partial \lambda} & = 0 \\
        %
        \frac{\partial^2 x^2}{\partial \lambda^2} +
        \Gamma^{2}_{\mu\nu}
        \frac{\partial x^\mu}{\partial \lambda}
        \frac{\partial x^\nu}{\partial \lambda} & = 0
    \end{align}
\end{subequations}
%
After plugging in \cref{eq:ass5_coeff} we obtain
%
\begin{subequations}
    \begin{align}
        \frac{\partial^2 \theta}{\partial \lambda^2} -
        \sin\theta\cos\theta
        \frac{\partial \phi}{\partial \lambda}
        \frac{\partial \phi}{\partial \lambda} & = 0 \\
        %
        \frac{\partial^2 \phi}{\partial \lambda^2} +
        2\frac{\cos\theta}{\sin\theta}
        \frac{\partial \theta}{\partial \lambda}
        \frac{\partial \phi}{\partial \lambda} & = 0
    \end{align}
\end{subequations}
%
Now we can set $\theta$ and $\phi$ as affine parameters

\subsection*{$\lambda \rightarrow \theta$}

\begin{subequations}
    \begin{align}
        -\sin\theta\cos\theta
        \frac{\partial \phi}{\partial \theta}
        \frac{\partial \phi}{\partial \theta} & = 0 \\
        %
        \frac{\partial^2 \phi}{\partial \theta^2} +
        2\frac{\cos\theta}{\sin\theta}
        \frac{\partial \phi}{\partial \theta} & = 0
    \end{align}
\end{subequations}
%
Solution to this set of equations is trivial namely
%
\begin{equation}
    \frac{\partial \phi}{\partial \theta} = 0
    \quad \Rightarrow \quad
    \phi = \text{const}
\end{equation}
%
It means that longitudinal lines are geodesics in this metric.

\subsection*{$\lambda \rightarrow \phi$}

\begin{subequations}
    \begin{align}
        \frac{\partial^2 \theta}{\partial \phi^2} -
        \sin\theta\cos\theta                  & = 0 \\
        %
        2\frac{\cos\theta}{\sin\theta}
        \frac{\partial \theta}{\partial \phi} & = 0
    \end{align}
\end{subequations}
%
From second equation we have that
%
\begin{equation}
    \frac{\partial \theta}{\partial \phi} = 0
\end{equation}
% 
but it does not solve first equation for every $\theta$. This system of
equations has solutions only when
%
\begin{equation}
    \sin\theta\cos\theta = 0
    \quad \Rightarrow \quad
    \theta = 0 ~\vee~ \theta = \frac{\pi}{2} ~\vee~ \theta = \pi
\end{equation}
%
since $\theta \in \left[0,\pi\right]$. But for $\theta = 0$ or $\theta = \pi$
geodesic line is just one point, because those are poles. Only $\theta =
    \frac{\pi}{2}$ gives non-trivial geodesic. This line is called equator.


\begin{figure}[ht]
    \centering
    \begin{tikzpicture}[tdplot_main_coords, scale = 3.5]
        \coordinate (P1) at ({0},{0},{1});
        \coordinate (P2) at ({0},{0},{-1});
        \coordinate (O) at (0,0,0);
        \pgfmathsetmacro{\rvec}{1.4}
        \pgfmathsetmacro{\thetavec}{45}
        \pgfmathsetmacro{\phivec}{60}

        \tdplotsetcoord{P}{\rvec}{\thetavec}{\phivec}

        \shade[ball color = lightgray, opacity = 0.5] (0,0,0) circle (1cm);

        \tdplotsetrotatedcoords{0}{0}{0};
        \draw[dashed, tdplot_rotated_coords, red] (0,0,0) circle (1);

        \tdplotsetrotatedcoords{-90}{90}{0};
        \draw[dashed, tdplot_rotated_coords, blue] (1,0,0) arc (0:180:1);

        \tdplotsetrotatedcoords{0}{90}{0};
        \draw[dashed, tdplot_rotated_coords, blue] (1,0,0) arc (0:180:1);

        \draw[dashed, gray] (0,0,0) -- (-1,0,0);
        \draw[dashed, gray] (0,0,0) -- (0,-1,0);

        \draw[-stealth] (0,0,0) -- (1.80,0,0) node[below left] {$x$};
        \draw[-stealth] (0,0,0) -- (0,1.30,0) node[below right] {$y$};
        \draw[-stealth] (0,0,0) -- (0,0,1.30) node[above] {$z$};
        \node[right] at (P1) {$P_1$};
        \node[right] at (P2) {$P_2$};

        \draw[-stealth, color=black] (O) -- (P) node[midway,below] {};

        \tdplotdrawarc{(O)}{0.2}{0}{\phivec}{anchor=north}{$\phi$}
        \draw[dashed, color=magenta] (O) -- (Pxy);
        \draw[dashed, color=magenta] (P) -- (Pxy);

        \tdplotsetthetaplanecoords{\phivec}
        \tdplotdrawarc[tdplot_rotated_coords]{(0,0,0)}{0.5}{\thetavec}{0}{anchor=south west}{$\theta$}

        \draw[fill = lightgray!50] (P1) circle (0.5pt);
        \draw[fill = lightgray!50] (P2) circle (0.5pt);
    \end{tikzpicture}
    \caption{Visualization of geodesics -- \textcolor{red}{red} line is equator,
        \textcolor{blue}{blue} lines are two possible meridians, $P_1$ and $P_2$ are poles.}
\end{figure}

\subproblem

\subsection*{$\theta = \frac{\pi}{2},~\phi = 0 \rightarrow \theta = 0,~ \phi = 0$}

We use equation of parallel transport of vector \bs{v} along curve $\gamma$ with
$\frac{\dd \gamma^\mu}{\dd \lambda} = w^\mu$
%
\begin{equation}
    \nabla_{\bs{w}} \bs{v} = 0
\end{equation}
%
We can write it explicitly
%
\begin{equation}
    w^\nu \partial_\nu v^\mu + w^\nu \Gamma^\mu_{\nu\sigma} v^\sigma = 0
\end{equation}
%
or substituting connections I've calculated before
%
\begin{subequations}
    \begin{align}
        w^1 \partial_1 v^1 + w^2 \partial_2 v^1 + w^2 \Gamma^1_{22} v^2                         & = 0 \\
        w^1 \partial_1 v^2 + w^2 \partial_2 v^2 + w^1 \Gamma^2_{12} v^2 + w^2 \Gamma^2_{21} v^1 & = 0
    \end{align}
\end{subequations}
%
If we move along meridian then we can take $\lambda \rightarrow \theta$. From
this we get coefficients $w$ namely $w^1 = \frac{\dd \theta}{\dd \theta} = 1$
and $w^2 = \frac{\dd \phi}{\dd \theta} = 0$. Putting this and connection
coefficients into equations we obtain
%
\begin{subequations}
    \begin{align}
        \partial_1 v^1                                     & = 0 \\
        \partial_1 v^2 + \frac{\cos\theta}{\sin\theta} v^2 & = 0
    \end{align}
\end{subequations}
%
Second equation we multiply by $\sin\theta$ and simplify
%
\begin{subequations}
    \begin{align}
        \partial_1 v^1                        & = 0 \\
        \partial_1 \left(v^2\sin\theta\right) & = 0
    \end{align}
\end{subequations}
%
This gives use
%
\begin{subequations}
    \begin{align}
        v^1 & = C_1                    \\
        v^2 & = \frac{C_2}{\sin\theta}
    \end{align}
\end{subequations}
%
Constants $C_1$ and $C_2$ depends on the vector we transport:
\begin{itemize}
    \item for $\frac{\partial}{\partial \theta}$ we have at the beginning
          ($\theta = \frac{\pi}{2}$) $v=(1,0)$ so $C_1 = 1$ and $C_2 = 0$ so the
          transport
          \begin{equation}
              \bs{v} = (1,0) \rightarrow (1,0) = \bs{v}'
          \end{equation}
          does not change this vector.
    \item for $\frac{\partial}{\partial \phi}$ we have at the beginning
          $v=(0,1)$ so $C_1 = 0$ and $C_2 = 1$ so the transport
          \begin{equation}
              \bs{v} = (0,1) \rightarrow (0,\frac{1}{\sin\theta}) = \bs{v}'
          \end{equation}
          is undefined at point $\theta = 0$
\end{itemize}

\subsection*{$\theta = \frac{\pi}{2},~\phi = 0 \rightarrow \theta = \frac{\pi}{2},~ \phi = \frac{\pi}{4}$}

%
If we move along equator then we can take $\lambda \rightarrow \phi$. From
this we get coefficients $w$ namely $w^1 = \frac{\dd \theta}{\dd \phi} = 0$
and $w^2 = \frac{\dd \phi}{\dd \phi} = 1$. Putting this and connection
coefficients into equations we obtain
%
\begin{subequations}
    \begin{align}
        \partial_2 v^1 - \sin\theta\cos\theta v^2          & = 0 \\
        \partial_2 v^2 + \frac{\cos\theta}{\sin\theta} v^1 & = 0
    \end{align}
\end{subequations}
%
which simplifies to
%
\begin{subequations}
    \begin{align}
        \partial_2 v^1 & = 0 \\
        \partial_2 v^2 & = 0
    \end{align}
\end{subequations}
%
because $\cos\theta \Big|_{\theta=\frac{\pi}{2}}= 0$
%
\begin{subequations}
    \begin{align}
        v^1 & = C_1 \\
        v^2 & = C_2
    \end{align}
\end{subequations}
%
Constants $C_1$ and $C_2$ depends on the vector we transport:
\begin{itemize}
    \item for $\frac{\partial}{\partial \theta}$ we have at the beginning
          ($\phi = 0$) $v=(1,0)$ so $C_1 = 1$ and $C_2 = 0$ so the
          transport
          \begin{equation}
              \bs{v} = (1,0) \rightarrow (1,0) = \bs{v}'
          \end{equation}
          does not change this vector.
    \item for $\frac{\partial}{\partial \phi}$ we have at the beginning
          $v=(0,1)$ so $C_1 = 0$ and $C_2 = 1$ so the transport
          \begin{equation}
              \bs{v} = (0,1) \rightarrow (0,1) = \bs{v}'
          \end{equation}
          does not change this vector.
\end{itemize}

\begin{figure}[ht]
    \centering
    \begin{tikzpicture}[tdplot_main_coords, scale = 2.5]
        \coordinate (O) at (0,0,0);

        \pgfmathsetmacro{\rvec}{1}
        \pgfmathsetmacro{\thetavec}{45}
        \pgfmathsetmacro{\phivec}{45}

        \pgfmathsetmacro{\sinphi}{sin(\phivec)}
        \pgfmathsetmacro{\cosphi}{cos(\phivec)}

        \pgfmathsetmacro{\sintheta}{sin(\thetavec)}
        \pgfmathsetmacro{\costheta}{cos(\thetavec)}

        \tdplotsetcoord{P}{\rvec}{\thetavec}{\phivec}

        \shade[ball color = lightgray, opacity = 0.5] (0,0,0) circle (1cm);

        \tdplotsetrotatedcoords{0}{0}{0};
        \draw[dashed, tdplot_rotated_coords, gray] (0,0,0) circle (1);

        \tdplotsetrotatedcoords{-90}{90}{0};
        \draw[dashed, tdplot_rotated_coords, gray] (1,0,0) arc (0:180:1);

        \tdplotsetrotatedcoords{0}{90}{0};
        \draw[dashed, tdplot_rotated_coords, gray] (1,0,0) arc (0:180:1);

        \draw[dashed, gray] (0,0,0) -- (-1,0,0);
        \draw[dashed, gray] (0,0,0) -- (0,-1,0);

        \draw[-stealth] (0,0,0) -- (1.80,0,0) node[below left] {$x$};
        \draw[-stealth] (0,0,0) -- (0,1.30,0) node[below right] {$y$};
        \draw[-stealth] (0,0,0) -- (0,0,1.30) node[above] {$z$};

        \draw[-stealth, color=red]
        (\sinphi,\cosphi,0) -- (\sinphi-\cosphi, \cosphi+\sinphi, 0) node[above]
        {$\frac{\partial}{\partial \phi}$};
        \draw[-stealth, color=red]
        (\sinphi,\cosphi,0) -- (\sinphi,\cosphi,-1) node[left]
        {$\frac{\partial}{\partial \theta}$};

        \draw[-stealth, color=blue]
        (\sintheta,0,\costheta) -- (\sintheta+\costheta,0,\costheta-\sintheta) node[left]
        {$\frac{\partial}{\partial \theta}$};
        \draw[-stealth, color=blue]
        (\sintheta,0,\costheta) -- (\sintheta,1,\costheta) node[above]
        {$\frac{\partial}{\partial \phi}$};

    \end{tikzpicture}
    \caption{Visualization of parallel transport -- \textcolor{red}{red} is
        transport along equator, \textcolor{blue}{blue} is transport along
        meridians.}
\end{figure}

\problem

Materic is given by
%
\begin{equation}
    g_{\nu\mu} =
    \begin{pmatrix}
        -B(r) & 0    & 0   & 0               \\
        0     & A(r) & 0   & 0               \\
        0     & 0    & r^2 & 0               \\
        0     & 0    & 0   & r^2\sin^2\theta
    \end{pmatrix}
\end{equation}
%
We can calculate connection coefficients using relation
%
\begin{equation}
    \Gamma^\mu_{\nu\sigma} =
    \frac{1}{2} g^{\mu\rho} \left(
    \partial_\nu g_{\rho\sigma} +
    \partial_\sigma g_{\rho\nu} -
    \partial_\rho g_{\nu\sigma}\right)
\end{equation}
%
Becasue metric is diagonal we can simplify this expression:
%
\begin{equation}
    \Gamma^\mu_{\nu\sigma} =
    \frac{1}{2} g^{\mu\mu} \left(
    \partial_\nu g_{\mu\sigma} +
    \partial_\sigma g_{\mu\nu} -
    \partial_\mu g_{\nu\sigma}\right)
\end{equation}
%
\begin{subequations}
    \begin{align}
        \Gamma^0 & =
        -\frac{1}{2} \frac{1}{B(r)}
        \begin{pmatrix}
            0       & \dot{B} & 0 & 0 \\
            \dot{B} & 0       & 0 & 0 \\
            0       & 0       & 0 & 0 \\
            0       & 0       & 0 & 0 \\
        \end{pmatrix} \\
        \Gamma^1 & =
        \frac{1}{2} \frac{1}{A(r)}
        \begin{pmatrix}
            \dot{B} & 0       & 0   & 0               \\
            0        & \dot{A} & 0   & 0               \\
            0        & 0       & -2r & 0               \\
            0        & 0       & 0   & -2r\sin^2\theta \\
        \end{pmatrix} \\
        \Gamma^2 & =
        \frac{1}{2} \frac{1}{r^2}
        \begin{pmatrix}
            0 & 0  & 0  & 0                         \\
            0 & 0  & 2r & 0                         \\
            0 & 2r & 0  & 0                         \\
            0 & 0  & 0  & -r^22\sin\theta\cos\theta \\
        \end{pmatrix} =
        \begin{pmatrix}
            0 & 0           & 0           & 0                     \\
            0 & 0           & \frac{1}{r} & 0                     \\
            0 & \frac{1}{r} & 0           & 0                     \\
            0 & 0           & 0           & -\sin\theta\cos\theta \\
        \end{pmatrix} \\
        \Gamma^3 & =
        \frac{1}{2} \frac{1}{r^2\sin^2\theta}
        \begin{pmatrix}
            0 & 0              & 0                        & 0                        \\
            0 & 0              & 0                        & 2r\sin^2\theta           \\
            0 & 0              & 0                        & r^22\sin\theta\cos\theta \\
            0 & 2r\sin^2\theta & r^22\sin\theta\cos\theta & 0                        \\
        \end{pmatrix} =
        \begin{pmatrix}
            0 & 0           & 0                             & 0                             \\
            0 & 0           & 0                             & \frac{1}{r}                   \\
            0 & 0           & 0                             & \frac{\cos\theta}{\sin\theta} \\
            0 & \frac{1}{r} & \frac{\cos\theta}{\sin\theta} & 0                             \\
        \end{pmatrix}
    \end{align}
\end{subequations}
%
Geodesic equation
%
\begin{equation}
    \frac{\partial^2 x^\sigma}{\partial \lambda^2} +
    \Gamma^\sigma_{\mu\nu}\frac{\partial x^\mu}{\partial \lambda}\frac{\partial x^\nu}{\partial \lambda} = 0
\end{equation}
%
\subsection*{$\sigma = 0$}

\begin{equation}
    \frac{\partial^2 t}{\partial \lambda^2} -
    \frac{1}{2} \frac{1}{B(t)}
    \begin{pmatrix}
        \frac{\partial t}{\partial \lambda}      &
        \frac{\partial r}{\partial \lambda}      &
        \frac{\partial \theta}{\partial \lambda} &
        \frac{\partial \phi}{\partial \lambda}
    \end{pmatrix}
    \begin{pmatrix}
        0       & \dot{B} & 0 & 0 \\
        \dot{B} & 0       & 0 & 0 \\
        0       & 0       & 0 & 0 \\
        0       & 0       & 0 & 0 \\
    \end{pmatrix}
    \begin{pmatrix}
        \frac{\partial t}{\partial \lambda}      \\[6pt]
        \frac{\partial r}{\partial \lambda}      \\[6pt]
        \frac{\partial \theta}{\partial \lambda} \\[6pt]
        \frac{\partial \phi}{\partial \lambda}
    \end{pmatrix} = 0
\end{equation}

\begin{equation}
    \frac{\partial^2 t}{\partial \lambda^2} -
    \frac{1}{2} \frac{1}{B(t)}
    \left(-2\dot{B} \frac{\partial t}{\partial \lambda}\frac{\partial r}{\partial \lambda}\right) = 0
\end{equation}

\begin{equation}
    \frac{\partial^2 t}{\partial \lambda^2} +
    \frac{\dot{B}}{B}
    \frac{\partial t}{\partial \lambda}\frac{\partial r}{\partial \lambda} = 0
\end{equation}

\subsection*{$\sigma = 1$}

\begin{equation}
    \frac{\partial^2 r}{\partial \lambda^2} +
    \frac{1}{2} \frac{1}{A(r)}
    \begin{pmatrix}
        \frac{\partial t}{\partial \lambda}      &
        \frac{\partial r}{\partial \lambda}      &
        \frac{\partial \theta}{\partial \lambda} &
        \frac{\partial \phi}{\partial \lambda}
    \end{pmatrix}
    \begin{pmatrix}
        \dot{B} & 0       & 0   & 0               \\
        0        & \dot{A} & 0   & 0               \\
        0        & 0       & -2r & 0               \\
        0        & 0       & 0   & -2r\sin^2\theta \\
    \end{pmatrix}
    \begin{pmatrix}
        \frac{\partial t}{\partial \lambda}      \\[6pt]
        \frac{\partial r}{\partial \lambda}      \\[6pt]
        \frac{\partial \theta}{\partial \lambda} \\[6pt]
        \frac{\partial \phi}{\partial \lambda}
    \end{pmatrix} = 0
\end{equation}

\begin{equation}
    \frac{\partial^2 r}{\partial \lambda^2} +
    \frac{\dot{B}}{2A}\left(\frac{\partial t}{\partial \lambda}\right)^2 +
    \frac{\dot{A}}{2A}\left(\frac{\partial r}{\partial \lambda}\right)^2 -
    \frac{r}{A}\left(\frac{\partial \theta}{\partial \lambda}\right)^2 -
    \frac{r\sin^2\theta}{A}\left(\frac{\partial \phi}{\partial \lambda}\right)^2 = 0
\end{equation}

\subsection*{$\sigma = 2$}

\begin{equation}
    \frac{\partial^2 \theta}{\partial \lambda^2} +
    \begin{pmatrix}
        \frac{\partial t}{\partial \lambda}      &
        \frac{\partial r}{\partial \lambda}      &
        \frac{\partial \theta}{\partial \lambda} &
        \frac{\partial \phi}{\partial \lambda}
    \end{pmatrix}
    \begin{pmatrix}
        0 & 0           & 0           & 0                     \\
        0 & 0           & \frac{1}{r} & 0                     \\
        0 & \frac{1}{r} & 0           & 0                     \\
        0 & 0           & 0           & -\sin\theta\cos\theta \\
    \end{pmatrix}
    \begin{pmatrix}
        \frac{\partial t}{\partial \lambda}      \\[6pt]
        \frac{\partial r}{\partial \lambda}      \\[6pt]
        \frac{\partial \theta}{\partial \lambda} \\[6pt]
        \frac{\partial \phi}{\partial \lambda}
    \end{pmatrix} = 0
\end{equation}

\begin{equation}
    \frac{\partial^2 \theta}{\partial \lambda^2} +
    \frac{2}{r} \frac{\partial r}{\partial \lambda}\frac{\partial \theta}{\partial \lambda} -
    \sin\theta\cos\theta \left(\frac{\partial \phi}{\partial \lambda}\right)^2 = 0
\end{equation}

\subsection*{$\sigma = 2$}

\begin{equation}
    \frac{\partial^2 \phi}{\partial \lambda^2} +
    \begin{pmatrix}
        \frac{\partial t}{\partial \lambda}      &
        \frac{\partial r}{\partial \lambda}      &
        \frac{\partial \theta}{\partial \lambda} &
        \frac{\partial \phi}{\partial \lambda}
    \end{pmatrix}
    \begin{pmatrix}
        0 & 0           & 0                             & 0                             \\
        0 & 0           & 0                             & \frac{1}{r}                   \\
        0 & 0           & 0                             & \frac{\cos\theta}{\sin\theta} \\
        0 & \frac{1}{r} & \frac{\cos\theta}{\sin\theta} & 0                             \\
    \end{pmatrix}
    \begin{pmatrix}
        \frac{\partial t}{\partial \lambda}      \\[6pt]
        \frac{\partial r}{\partial \lambda}      \\[6pt]
        \frac{\partial \theta}{\partial \lambda} \\[6pt]
        \frac{\partial \phi}{\partial \lambda}
    \end{pmatrix} = 0
\end{equation}

\begin{equation}
    \frac{\partial^2 \phi}{\partial \lambda^2} +
    \frac{2}{r} \frac{\partial r}{\partial \lambda}\frac{\partial \phi}{\partial \lambda} +
    2\frac{\cos\theta}{\sin\theta} \frac{\partial \theta}{\partial \lambda}\frac{\partial \phi}{\partial \lambda} = 0
\end{equation}
%
Gathering all equations
%

\begin{subequations}
    \begin{align}
        0 & = \frac{\partial^2 t}{\partial \lambda^2} +
        \frac{\dot{B}}{B}
        \frac{\partial t}{\partial \lambda}\frac{\partial r}{\partial \lambda}       \\
        %
        0 & = \frac{\partial^2 r}{\partial \lambda^2} +
        \frac{\dot{B}}{2A}\left(\frac{\partial t}{\partial \lambda}\right)^2 +
        \frac{\dot{A}}{2A}\left(\frac{\partial r}{\partial \lambda}\right)^2 -
        \frac{r}{A}\left(\frac{\partial \theta}{\partial \lambda}\right)^2 -
        \frac{r\sin^2\theta}{A}\left(\frac{\partial \phi}{\partial \lambda}\right)^2 \\
        %
        0 & = \frac{\partial^2 \theta}{\partial \lambda^2} +
        \frac{2}{r} \frac{\partial r}{\partial \lambda}\frac{\partial \theta}{\partial \lambda} -
        \sin\theta\cos\theta \left(\frac{\partial \phi}{\partial \lambda}\right)^2   \\
        %
        0 & = \frac{\partial^2 \phi}{\partial \lambda^2} +
        \frac{2}{r} \frac{\partial r}{\partial \lambda}\frac{\partial \phi}{\partial \lambda} +
        2\frac{\cos\theta}{\sin\theta} \frac{\partial \theta}{\partial \lambda}\frac{\partial \phi}{\partial \lambda}
    \end{align}
\end{subequations}

\problem

Metric is given by
%
\begin{equation}
    \bs{g} = - c^2 \left(1-2\frac{GM}{rc^2}\right) \dd t \otimes \dd t  + \dd \bs{x} \otimes \dd \bs{x}
\end{equation}
%
We can calculate infinitesimal interval of two events in this metric
%
\begin{equation}
    \dd s^2 = - c^2 \left(1-2\frac{GM}{rc^2}\right) \dd t^2 + \dd \bs{x}^2
\end{equation}
%
Simplifying this and assuming that the person has not been moving for whole year
we can write
%
\begin{equation}
    \dd s^2 = \dd t^2 \left(- c^2 \left(1-2\frac{GM}{rc^2}\right)  +
    \frac{\dd \bs{x}^2}{\dd t^2}\right) =
    - \dd t^2c^2 \left(1-2\frac{GM}{rc^2}\right)
\end{equation}
%
We can now calculate the proper time
%
\begin{equation}
    c^2 \dd \tau^2 = \dd t^2c^2 \left(1-2\frac{GM}{rc^2}\right)
    \quad \Rightarrow \quad
    \dd \tau = \dd t \sqrt{1-2\frac{GM}{rc^2}}
\end{equation}
%
and finite version
%
\begin{equation}
    \Delta \tau = \Delta t \sqrt{1-2\frac{GM}{rc^2}}
\end{equation}
%
Now I take $\Delta \tau_{0}$ to be proper time of person's feet and
$\Delta \tau_{H}$ for head (in distance $H$ from feet). Now I can calculate the
difference
%
\begin{equation}
    \Delta \tau_{H} - \Delta \tau_{0} =
    \Delta t \left(\sqrt{1-2\frac{GM}{(r+H)c^2}} - \sqrt{1-2\frac{GM}{rc^2}}\right)
\end{equation}
%
Because expressions under square are small I can Taylor expand them (using
$\sqrt{1-2x} \simeq 1 -x$)
%
\begin{equation}
    \Delta \tau_{H} - \Delta \tau_{0} =
    \Delta t \left(-\frac{GM}{(r+H)c^2} + \frac{GM}{rc^2} \right) =
    \Delta t \frac{GM}{c^2} \left(\frac{1}{r} - \frac{1}{r+H}\right) =
    \Delta t \frac{GM}{c^2r} \frac{H}{r+H}
\end{equation}
%
Plugging all the constants and assuming that person has height 1.8 m
after 1 year time difference is
%
\begin{equation}
    \Delta \tau_{H} - \Delta \tau_{0} = 6.3 \times 10^{-9}~\text{s}
\end{equation}

\begin{figure}[H]
    \centering
    \begin{tikzpicture}[domain=0:2]
        \pgfmathsetmacro{\C}{6.96e-1}
        \pgfmathsetmacro{\R}{6371000}
        \pgfmathsetmacro{\dt}{31557600}
        \begin{axis}
            [
            axis lines  = left,
            xlabel={$x$~[m]},
            ylabel={$\Delta t$~[ns]},
            xmin=0,
            xmax=2.4,
            ymin=0,
            ymax=7
            ]
            \addplot[very thick,blue] plot (\x, {\C*\dt*x/(\R+x)});
        \end{axis}
    \end{tikzpicture}
    \caption{Time difference between head and feet after one year with respect
        to height of a person}
    % \label{fig:zad4}
\end{figure}